\section{Enerji}

İş, güç yada enerji aynı anlamlıdır ve kuvvet doğrultusunda alınan yoldur. Aşağıdaki şekilde ifade edilebilir.

\begin{equation}
    W = \int_{C} \vec{F}\text{.d}\vec{r} \qquad \qquad \parbox{4cm}{\footnotesize$\begin{aligned}
        W\text{: İş (j)} \\
        C\text{: Yol (m)} \\
        F\text{: Kuvvet (N)}
\end{aligned}$}
\end{equation}

\subsubsection*{Bir Doğru Üzerinde Alınan Yol İçin İş İlişkisi}
\begin{equation}
    W = F \Delta x  \qquad \qquad \parbox{4cm}{\footnotesize$\begin{aligned}
        W\text{: İş (j)} \\
        F\text{: Alınan Yol Doğrultusunda} \\ \text{Uygulanan Kuvvet (N)} \\
        \Delta x\text{: Alınan Yol (m)}
\end{aligned}$}
\end{equation}

\subsubsection*{Kütle Potansiyel Enerji İlişkisi}

\begin{equation}
    E_p = mgh \qquad \qquad \parbox{4cm}{\footnotesize$\begin{aligned}
        E_p\text{: Potansiyel Enerji (j) } \\
        m\text{: Kütle (kg) } \\
        g\text{: Yerçekimi İvmesi } (\frac{m}{s^2}) \\ 
        h\text{: Yükseklik (m) }
    \end{aligned}$}
\end{equation}

\subsubsection*{Kütle Kinetik Enerji Hız İlişkisi}

\begin{equation}
    E_k = \frac{1}{2} m v^2 \qquad \qquad \parbox{4cm}{\footnotesize$\begin{aligned}
        E_k\text{: Kinetik  Enerji (j) } \\
        m\text{: Kütle (kg) } \\ 
        v\text{: Hız } (\frac{m}{s})
    \end{aligned}$}
\end{equation}

\subsubsection*{Dönen Cisimler İçin Kinetik Enerji Hız İlişkisi}

\begin{equation}
    E_k = \frac{1}{2} I w^2 \qquad \qquad \parbox{4cm}{\footnotesize$\begin{aligned}
        E_k\text{: Kinetik  Enerji (j) } \\
        I\text{: Eylemsizlik Momenti (} kg*m^2\text{)} \\
        w\text{: Açısal Hız } (\frac{\text{radyan}}{s})
    \end{aligned}$}
\end{equation}

\subsubsection*{Esneklik Potansiyel Enerjisi}
\begin{equation}
    E_p = \frac{1}{2}k(\Delta x)^2 \qquad \qquad \parbox{4cm}{\footnotesize$\begin{aligned}
        E_p\text{: Esneklik Potansiyel Enerjisi (j) } \\
        k\text{: Yay Sabiti} \\
        \Delta x\text{: Yayın Uzama Miktarı (m)}
\end{aligned}$}
\end{equation}

\subsubsection*{Elektrik Devrelerinde Harcanan Enerji}
\begin{equation}
    E = VIt = I^2Rt = \frac{V^2}{R}t \qquad \qquad \parbox{4cm}{\footnotesize$\begin{aligned}
        E\text{: Enerji (j) } \\
        V\text{: Potansiyel Fark } (V) \\
        I\text{: Akım Şiddeti } (A) \\
        t\text{: Zaman (s)} \\
        R\text{: Direnç } (\Omega)
\end{aligned}$}
\end{equation}

\subsubsection*{Elektriksel Yükler Arası Potansiyel Enerji Formulü}
\begin{equation}
    E = k \frac{q_1 q_2}{d} \qquad \qquad \parbox{4cm}{\footnotesize$\begin{aligned}
        E\text{: Elektriksel Potansiyel Enerji} (j) \\
        k\text{: Coulumb Sabiti } \\
        q_1\text{, }g_2\text{: Cismlerin Yük Miktarı } (C) \\
        d\text{: Mesafe (m) }
\end{aligned}$}
\end{equation}

\subsubsection*{Elektrik Alana Karşı Yapılan İş}
\begin{equation}
    W = q \Delta V = q (V_s - V_i) \qquad \qquad \parbox{4cm}{\footnotesize$\begin{aligned}
        W\text{: İş (j)} \\
        q\text{: Cismin Yük Miktarı } (C) \\
        \Delta V\text{: Elektriksel Potansiyel Değişimi } (V) \\
        V_s\text{: Son Elektriksel Potansiyel (V) } \\
        V_i\text{: İlk Elektriksel Potansiyel (V) }
\end{aligned}$}
\end{equation}

\subsubsection*{Düzgün Elektrik Alanda Yol Alan Parçacığın Yaptığı İş}
\begin{equation}
    W = qEx\qquad \qquad \parbox{4cm}{\footnotesize$\begin{aligned}
        W\text{: İş (j)} \\
        E\text{: Elektrik Alan } \frac{N}{C} \\
        q\text{: Cismin Yük Miktarı } (C) \\
        x\text{: Alınan Yol (m) }
\end{aligned}$}
\end{equation}

\subimport{./}{power.tex}
