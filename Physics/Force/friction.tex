\subsection{Sürtünme Kuvveti}

\subsubsection*{Katılarda Sürtünme Kuvveti}

\begin{equation}
    \vec{F_s} = -kN\hat{v} \qquad \qquad \parbox{4cm}{\footnotesize$\begin{aligned}
        \vec{F_s}\text{: Sürtünme Kuvveti (N)} \\
        k\text{: Sürtünme Katsayısı} \\ 
        N\text{: Yerin Tepkisi (N)} \\
        \hat{v}\text{: Normalleştirilmiş Hız } (\frac{m}{s}) 
\end{aligned}$}
\end{equation}

\subsubsection*{Akışkanlarda Sürtünme Kuvveti}
\begin{equation}
    \vec{F_s} = -\frac{1}{2}\rho \vec{v}^2 A C_d \hat{v} \qquad \qquad \parbox{5.5cm}{\footnotesize$\begin{aligned}
        \vec{F_s}\text{: Sürtünme Kuvveti (N)} \\
        \rho\text{: Akışkanın Özkütlesi } (\frac{kg}{m^3} = \frac{g}{dm^3} = \frac{mg}{cm^3})  \\
        \vec{v}\text{: Hız } (\frac{m}{s}) \\
        A\text{: Yüzey Alanı } (m^2) \\
        C_d\text{: Direnç Katsayısı} \\
        \hat{v}\text{: Normalleştirilmiş Hız } (\frac{m}{s}) 
\end{aligned}$}
\end{equation}

Küçük hızlarda için formül,

\begin{equation}
    \vec{F_s} = k A \vec{v}^2 \hat{v} \qquad \qquad \parbox{4cm}{\footnotesize$\begin{aligned}
        \vec{F_s}\text{: Sürtünme Kuvveti (N)} \\
        k\text{: Sürtünme Katsayısı} \\ 
        A\text{: Yüzey Alanı } (m^2) \\
        \vec{v}\text{: Hız } (\frac{m}{s}) \\
        \hat{v}\text{: Normalleştirilmiş Hız } (\frac{m}{s}) 
\end{aligned}$}
\end{equation}

şeklinde yazılabilir.