\subsection{Manyetik Kuvvet}

\subsubsection*{Manyetik Çekim Kuvveti}
\begin{equation}
    F = K\frac{q_1 q_2}{d^2} \qquad \qquad \parbox{4cm}{\footnotesize$\begin{aligned}
        F\text{: Kuvvet (N) } \\
        K\text{: Manyetik Alan Sabiti } \\
        q_1\text{, }q_2\text{: Cismin Yük Miktarı } (C) \\
        d\text{: Cisimler Arası Mesafe (m) }
\end{aligned}$}
\end{equation}

\subsubsection*{Manyetik Alanın Şiddeti}
\begin{equation}
    B = 2\pi K \frac{I}{d} \qquad \qquad \parbox{4cm}{\footnotesize$\begin{aligned}
        B\text{: Manyetik Akı Yoğunluğu (T) } \\
        K\text{: Manyetik Alan Sabiti } \\
        I\text{: Akım } (A) \\
        d\text{: Mesafe (m) }
\end{aligned}$}
\end{equation}

\subsubsection*{Bobinler Için Manyetik Alanın Şiddeti}
\begin{equation}
    B = 4\pi K \Pi N \frac{I}{l} \qquad \qquad \parbox{4cm}{\footnotesize$\begin{aligned}
        B\text{: Manyetik Akı Yoğunluğu (T) } \\
        K\text{: Manyetik Alan Sabiti } \\
        N\text{: Sarım Sayısı } \\
        I\text{: Akım } (A) \\
        l\text{: Uzunluk (m) }
\end{aligned}$}
\end{equation}
