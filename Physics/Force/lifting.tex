\documentclass[./lifting.tex]{subfiles}
\subsection{Kaldırma Kuvveti}

\subsubsection*{Akışkanlarda Yüzen Cisimler için Kaldırma Kuvveti}
\begin{equation}
    F_{k} = V_{b} d_{a} g \qquad \qquad \parbox{5.5cm}{\footnotesize$\begin{aligned}
        F_{k}\text{: Kaldırma Kuvveti (N)} \\
        V_{b}\text{: Hacim } (m^3) \\
        d_{a}\text{: Akışkanın Özkütlesi } (\frac{kg}{m^3} = \frac{g}{dm^3} = \frac{mg}{cm^3})  \\
        g\text{: Yerçekimi İvmesi} (\frac{m}{s^2})
\end{aligned}$}
\end{equation}

\subsubsection*{Akışkankarda Batan Cisimler için Kaldırma Kuvveti}
\begin{equation}
    F_{k} = V_{b} d_{a} g - N\qquad \qquad \parbox{5.5cm}{\footnotesize$\begin{aligned}
        F_{k}\text{: Kaldırma Kuvveti (N)} \\
        V_{b}\text{: Hacim } (m^3) \\
        d_{a}\text{: Akışkanın Özkütlesi } (\frac{kg}{m^3} = \frac{g}{dm^3} = \frac{mg}{cm^3})  \\
        g\text{: Yerçekimi İvmesi} (\frac{m}{s^2}) \\
        N\text{: Yerin Tepkisi (N) }
\end{aligned}$}
\end{equation}

\subsubsection*{Arşimet Prensibi}
\begin{theorem}
    Taşan sıvının ağırlığı, cisme uyguladığı kaldırma kuvvetine eşittir.
\end{theorem}
