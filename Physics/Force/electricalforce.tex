\subsection{Elektriksel Kuvvet}

\subsubsection*{Coulomb Yasası}
\begin{equation}
    F = k\frac{q_1 q_2}{d^2} \qquad \qquad \parbox{4cm}{\footnotesize$\begin{aligned}
        F\text{: Kuvvet (N) } \\
        k\text{: Coulumb Sabiti } \\
        q_1\text{, }q_2\text{: Cismin Yük Miktarı } (C) \\
        d\text{: Cisimler Arası Mesafe (m) }
\end{aligned}$}
\end{equation}

\subsubsection*{Elektrik Alanın Kuvveti}
\begin{equation}
    E = k\frac{q}{d^2} \qquad \qquad \parbox{4cm}{\footnotesize$\begin{aligned}
        E\text{: Kuvvet } \frac{N}{C} \\
        k\text{: Coulumb Sabiti } \\
        q\text{: Cismin Yük Miktarı } (C) \\
        d\text{: Mesafe (m) }
\end{aligned}$}
\end{equation}

\subsubsection*{Parelel Plakalar Arası Elektrik Alanın Kuvveti}
\begin{equation}
    E = \frac{V}{d} \qquad \qquad \parbox{4cm}{\footnotesize$\begin{aligned}
        E\text{: Kuvvet } \frac{N}{C} \\
        V\text{: Üreteç Potansiyeli} (V) \\
        d\text{: Mesafe (m) }
\end{aligned}$}
\end{equation}