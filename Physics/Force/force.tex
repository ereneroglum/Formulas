\documentclass[../pyhsics.tex]{subfiles}
\section{Kuvvet}

Kuvvet,

\begin{equation}
    \vec{F} = \frac{d\vec{p}}{dt} \qquad \qquad \parbox{4cm}{\footnotesize$
    \begin{aligned}
        \vec{F}\text{: Kuvvet (N) } \\ 
        \vec{p}\text{: Momentum } (\frac{kg*m}{s} = N*s) \\
        t\text{: Zaman } (s)
    \end{aligned}$
    }
\end{equation}
şeklinde ifade edilebilir. 

\subsubsection*{Etki Tepki Kuvveti}
Tepki kuvveti, etki kuvvetiyle aynı büyüklükte, ters yönlüdür.
\begin{equation}
    \vec{T} = -\vec{F} \qquad \qquad \parbox{4cm}{\footnotesize$\begin{aligned}
        \vec{T}\text{: Tepki Kuvveti (N) } \\
        \vec{F}\text{: Etki Kuvveti (N) }
\end{aligned}$}
\end{equation}

\subsubsection*{Kuvvet İvme İlişkisi}

$\vec{P} = m\vec{v}$ alınırsa ve kütle zamanla değişmiyorsa,

\begin{equation}
    \vec{F} = m\vec{a} \qquad \qquad \parbox{4cm}{\footnotesize$\begin{aligned}
        \vec{F}\text{: Kuvvet (N) } \\
        m\text{: Kütle (kg) } \\
        \vec{a}\text{: İvme } (\frac{m}{s^2})
    \end{aligned}$}
\end{equation}

formülü yazılabilir.

\subsubsection*{Sıvılarda Basınç Kuvveti İlişkisi}
\begin{equation}
    F = hdgA \qquad \qquad \parbox{4cm}{\footnotesize$\begin{aligned}
        F\text{: Kuvvet (N) } \\
        h\text{: Yükseklik (m) } \\
        d\text{: Sıvının Özkütlesi } (\frac{kg}{m^3} = \frac{g}{dm^3} = \frac{mg}{cm^3})  \\
        g\text{: Yerçekimi İvmesi} (\frac{m}{s^2}) \\
        A\text{: Yüzey Alanı } (m^2)
\end{aligned}$}
\end{equation}

\subsubsection*{Hooke Yasası}
\begin{equation}
    \vec{F} = -k\vec{x} \qquad \qquad \parbox{4cm}{\footnotesize$\begin{aligned}
        \vec{F}\text{: Kuvvet (N) } \\
        k\text{: Yay Sabiti} \\
        \vec{x}\text{: Yayın Uzama Miktarı (m)}
\end{aligned}$}
\end{equation}

\subsubsection*{Ağırlık Kütle İlişkisi}
\begin{equation}
    \vec{G} = m\vec{g} \qquad \qquad \parbox{4cm}{\footnotesize$\begin{aligned}
        \vec{G}\text{: Ağırlık (N)} \\
        m\text{: Kütle (kg) } \\
        \vec{g}\text{: Yerçekimi İvmesi } (\frac{m}{s^2}) 
\end{aligned}$}
\end{equation}

\subfile{./friction.tex}
\subfile{./lifting.tex}
\subfile{./gravitationelforce.tex}
\subfile{./electricalforce.tex}
\subfile{./magneticforce.tex}