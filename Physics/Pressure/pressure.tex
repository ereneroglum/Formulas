\documentclass[../physics.tex]{subfiles}
\section{Basınç}

\subsubsection*{Kuvvet Basınç İlişkisi}
\begin{equation}
    P = \frac{F}{A} \qquad \qquad \parbox{4cm}{\footnotesize$\begin{aligned}
        P\text{: Basınç (Pa} = \frac{N}{m^2}\text{)} \\ 
        F\text{: Yüzeye Dik Kuvvet (N)} \\
        A\text{: Yüzey Alanı } (m^2)
\end{aligned}$}
\end{equation}

\subsubsection*{Sıvıların Kap Tabanına Yaptığı Basınç İlişkisi}
\begin{equation}
    P = hdg \qquad \qquad \parbox{4cm}{\footnotesize$\begin{aligned}
        P\text{: Basınç (Pa} = \frac{N}{m^2}\text{)} \\ 
        h\text{: Yükseklik (m) } \\
        d\text{: Özkütle } (\frac{kg}{m^3} = \frac{g}{dm^3} = \frac{mg}{cm^3})  \\
        g\text{: Yerçekimi İvmesi} (\frac{m}{s^2})
\end{aligned}$}
\end{equation}