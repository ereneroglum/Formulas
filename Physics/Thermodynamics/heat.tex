\documentclass[./thermodynamics.tex]{subfiles}
\subsection{Isı}

\subsubsection*{Bir Cismin Sıcaklığını Arttırmak İçin Gerekli Isı Formülü}
\begin{equation}
    Q = mc\Delta T = C\Delta T \qquad \qquad \parbox{4cm}{\footnotesize$\begin{aligned}
        Q\text{: Isı (j) }  \\
        m\text{: Kütle (kg) } \\
        c\text{: Öz Isı } (\frac{j}{kg\text{\textdegree K}}) \\
        \Delta T \text{: Sıcaklık Değişimi (\textdegree K)} \\
        C\text{: Isı Sığası } (\frac{j}{\text{\textdegree K}})
\end{aligned}$}
\end{equation}

\subsubsection*{Hal Değişimi İçin Gerekli Isı Formülü}
\begin{equation}
    Q = mL \qquad \qquad \parbox{4cm}{\footnotesize$\begin{aligned}
        Q\text{: Isı (j) }  \\
        m\text{: Kütle (kg) } \\
        L\text{: Hal Değiştirme Isısı } (\frac{j}{kg})
\end{aligned}$}
\end{equation}
