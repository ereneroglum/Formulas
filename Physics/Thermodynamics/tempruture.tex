\documentclass[./thermodynamics.tex]{subfiles}
\subsection{Sıcaklık}

\subsubsection*{Derece Dönüşüm Formülleri}
\begin{equation}
    \frac{\text{\textdegree C}}{100} = \frac{\text{\textdegree F} - 32}{180} = \frac{\text{\textdegree K} - 273}{100} = \frac{\text{\textdegree X} - T_D}{T_K - T_D} \qquad \qquad \parbox{7cm}{\footnotesize$\begin{aligned}
        C\text{: Celcius Derece } \\
        F\text{: Fahreneit Derece } \\
        K\text{: Kelvin Derece } \\
        X\text{: Herhangi Bir X Termometresi } \\
        T_D\text{: Suyun X Termometresi için Donma Sıcaklığı } \\
        T_K\text{: Suyun X Termometresi için Kaynama Sıcaklığı }
\end{aligned}$}
\end{equation}
