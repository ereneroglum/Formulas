\subsection{Çizgisel Moment}

\subsubsection*{Çizgisel Moment Formulü}
\begin{equation}
    \vec{P} = m\vec{v} \qquad \qquad \parbox{4cm}{\footnotesize$\begin{aligned}
        \vec{P}\text{: Moment } \frac{\text{kg} m }{s} \\
        m\text{: Kütle (kg) } \\
        \vec{v}\text{: Hız } (\frac{m}{s})
\end{aligned}$}
\end{equation}

\subsubsection*{Itme Formulü}
\begin{equation}
    I = \Delta P = F\Delta t = m \Delta v \qquad \qquad \parbox{4cm}{\footnotesize$\begin{aligned}
        \vec{I}\text{: İtme } \frac{\text{kg} m }{s}
        \Delta P \text{: Moment } \frac{\text{kg} m }{s} \\
        F\text{: Kuvvet (N) } \\
        t\text{: Zaman (s)} \\
        m\text{: Kütle (kg) } \\
        \Delta v \text{: Hız } (\frac{m}{s})
\end{aligned}$}
\end{equation}

\subsubsection*{Esnek Çarpışan Cisimler İçin Hızın Korunumu}
\begin{equation}
    \vec{v_1}  + \vec{v_1} \prime = \vec{v_2} + \vec{v_2} \prime \qquad \qquad \parbox{4cm}{\footnotesize$\begin{aligned}
        \vec{v_1}\text{1. Cismin Çarpışma Öncesi Hızı } (\frac{m}{s}) \\
        \vec{v_1}\prime\text{1. Cismin Çarpışma Sonrası Hızı } (\frac{m}{s}) \\
        \vec{v_2}\text{2. Cismin Çarpışma Öncesi Hızı } (\frac{m}{s}) \\
        \vec{v_2}\prime\text{2. Cismin Çarpışma Sonrası Hızı } (\frac{m}{s})
\end{aligned}$}
\end{equation}