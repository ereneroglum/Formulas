\documentclass[./optik]{subfiles}
\subsection{Yansıma}

\begin{figure}[h!]
     \centering
     \begin{tikzpicture}
          \tkzDefPoint(0,0){A}
          \tkzDefPoint(8,0){B}
          \tkzDrawSegment(A,B)
          \tkzDefMidPoint(A,B) \tkzGetPoint{D}
          \tkzDefPoint(4,3){E}
          \tkzDrawSegment[dashed](E,D)
          \tkzLabelPoint(E){N}
          \tkzMarkRightAngle(E,D,B)
          \tkzLabelAngle[pos=-0.15](E,D,B){$\cdot$}
     
          \tkzDefPoint(2,2.7){G}
          \tkzDefPoint(6,2.7){H}
     
          \begin{scope}[decoration={
               markings,
               mark=at position 0.5 with {\arrow{>}}}
               ] 
               \tkzDrawSegments[postaction={decorate}](G,D)
               \tkzDrawSegments[postaction={decorate}](D,H)
           \end{scope}
     
           \tkzMarkAngle[mark=none, size=0.4](E,D,G)
           \tkzLabelAngle[pos=0.7](E,D,G){$\alpha$}
           \tkzMarkAngle[mark=none, size=0.4](H,D,E)
           \tkzLabelAngle[pos=0.7](H,D,E){$\alpha$}
     \end{tikzpicture}
\end{figure}

\subsubsection*{Yansımanın Temel Prensibi}

\begin{theorem}
    Işınların gelme açısı ile yansıma açısı birbirine eşittir.
\end{theorem}
