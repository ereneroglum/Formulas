\section{Madde}

\subsubsection*{Kütle Özkütle Hacim İlişkisi}

\begin{equation}
    d = \frac{m}{V} \qquad \qquad \parbox{4cm}{\footnotesize$
    \begin{aligned}
        d\text{: Özkütle } (\frac{kg}{m^3} = \frac{g}{dm^3} = \frac{mg}{cm^3}) \\
        m\text{: Kütle } (kg) \\
        V\text{: Hacim } (m^3)
    \end{aligned}$}
\end{equation}

\subsubsection*{Dayanıklılık Formulü}

\begin{equation}
    \text{Dayanıklılık} = \frac{A}{V} \qquad \qquad \parbox{4cm}{\footnotesize$
      \begin{aligned}
        A\text{: Kesit Alanı } (m^2) \\
        V\text{: Hacim } (m^3)
    \end{aligned}$}
\end{equation}

\subsubsection*{Prizmalar İçin Dayanıklılık Formulü}

\begin{equation}
    \text{Dayanıklılık} = \frac{1}{h} \qquad \qquad \parbox{4cm}{\footnotesize$
      \begin{aligned}
        h\text{: Yükseklik } (m)
    \end{aligned}$}
\end{equation}

\subsubsection*{Küreler İçin Dayanıklılık Formulü}

\begin{equation}
    \text{Dayanıklılık} = \frac{3}{4r} \qquad \qquad \parbox{4cm}{\footnotesize$
      \begin{aligned}
        r\text{: Yarıçap } (m)
    \end{aligned}$}
\end{equation}
