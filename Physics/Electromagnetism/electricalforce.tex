\subsection{Elektriksel Kuvvet}

\subsubsection*{Elektrik Alanın Kuvveti}
\begin{equation}
    E = k\frac{q}{d^2} \qquad \qquad \parbox{4cm}{\footnotesize$\begin{aligned}
        E\text{: Elektrik Alab } \frac{N}{C} \\
        k\text{: Coulumb Sabiti } \\
        q\text{: Cismin Yük Miktarı } (C) \\
        d\text{: Mesafe (m) }
\end{aligned}$}
\end{equation}

\subsubsection*{Elektriksel Yükler Arası Potansiyel Enerji Formulü}
\begin{equation}
    E = k \frac{q_1 q_2}{d} \qquad \qquad \parbox{4cm}{\footnotesize$\begin{aligned}
        E\text{: Elektriksel Potansiyel Enerji } (j) \\
        k\text{: Coulumb Sabiti } \\
        q_1\text{, }g_2\text{: Cismlerin Yük Miktarı } (C) \\
        d\text{: Mesafe (m) }
\end{aligned}$}
\end{equation}

\subsubsection*{Coulomb Yasası}
\begin{equation}
    F = k\frac{q_1 q_2}{d^2} \qquad \qquad \parbox{4cm}{\footnotesize$\begin{aligned}
        F\text{: Kuvvet (N) } \\
        k\text{: Coulumb Sabiti } \\
        q_1\text{, }q_2\text{: Cismin Yük Miktarı } (C) \\
        d\text{: Cisimler Arası Mesafe (m) }
\end{aligned}$}
\end{equation}

\subsubsection*{Noktasal Yükün Elektriksel Potansiyeli}
\begin{equation}
    V = k \frac{q}{d} \qquad \qquad \parbox{4cm}{\footnotesize$\begin{aligned}
        V\text{: Elektriksel Potansiyel } (V) \\
        k\text{: Coulumb Sabiti } \\
        q\text{: Cismin Yük Miktarı } (C) \\
        d\text{: Mesafe (m) }
\end{aligned}$}
\end{equation}

\subsubsection*{Parelel Plakalar Arası Elektrik Alanın Kuvveti}
\begin{equation}
    \vec{E} = \frac{V}{d} \qquad \qquad \parbox{4cm}{\footnotesize$\begin{aligned}
        \vec{E}\text{: Elektrik Alan } \frac{N}{C} \\
        V\text{: Üreteç Potansiyeli } (V) \\
        d\text{: Mesafe (m) }
\end{aligned}$}
\end{equation}

\subsubsection*{Parelel Plakalar Arası Parçacığa Etki Eden Kuvvet}
\begin{equation}
    \vec{F} = q\vec{E} \qquad \qquad \parbox{4cm}{\footnotesize$\begin{aligned}
        \vec{F}\text{: Kuvvet (N) } \\
        q\text{: Cismin Yük Miktarı } (C) \\
        \vec{E}\text{: Elektrik Alan } \frac{N}{C}
\end{aligned}$}
\end{equation}

\subsubsection*{Parelel Plakalar Arası Hareket Eden Parçacığın İvmesi}
\begin{equation}
    a = \frac{qV}{dm} \qquad \qquad \parbox{4cm}{\footnotesize$\begin{aligned}
        a\text{: İvme } (\frac{m}{s^2}) \\
        q\text{: Cismin Yük Miktarı } (C) \\
        V\text{: Üreteç Potansiyeli } (V) \\
        d\text{: Mesafe (m) } \\
        m\text{: Kütle (kg) }
\end{aligned}$}
\end{equation}

\subsubsection*{Elektrik Alana Karşı Yapılan İş}
\begin{equation}
    W = q \Delta V = q (V_s - V_i) \qquad \qquad \parbox{4cm}{\footnotesize$\begin{aligned}
        W\text{: İş (j) } \\
        q\text{: Cismin Yük Miktarı } (C) \\
        \Delta V\text{: Elektriksel Potansiyel Değişimi } (V) \\
        V_s\text{: Son Elektriksel Potansiyel (V) } \\
        V_i\text{: İlk Elektriksel Potansiyel (V) }
\end{aligned}$}
\end{equation}

\subsubsection*{Düzgün Elektrik Alanda Yol Alan Parçacığın Yaptığı İş}
\begin{equation}
    W = q E x\qquad \qquad \parbox{4cm}{\footnotesize$\begin{aligned}
        W\text{: İş (j)} \\
        E\text{: Elektrik Alan } \frac{N}{C} \\
        q\text{: Cismin Yük Miktarı } (C) \\
        x\text{: Alınan Yol (m) }
\end{aligned}$}
\end{equation}
