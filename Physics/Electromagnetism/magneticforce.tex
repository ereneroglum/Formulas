\subsection{Manyetik Kuvvet}

\subsubsection*{Manyetik Çekim Kuvveti}
\begin{equation}
    F = K\frac{q_1 q_2}{d^2} \qquad \qquad \parbox{4cm}{\footnotesize$\begin{aligned}
        F\text{: Kuvvet (N) } \\
        K\text{: Manyetik Alan Sabiti } \\
        q_1\text{, }q_2\text{: Cismin Yük Miktarı } (C) \\
        d\text{: Cisimler Arası Mesafe (m) }
\end{aligned}$}
\end{equation}

\subsubsection*{Manyetik Alanın Şiddeti}
\begin{equation}
    B = 2 K \frac{I}{d} \qquad \qquad \parbox{4cm}{\footnotesize$\begin{aligned}
        B\text{: Manyetik Alan (T) } \\
        K\text{: Manyetik Alan Sabiti } \\
        I\text{: Akım } (A) \\
        d\text{: Mesafe (m) }
\end{aligned}$}
\end{equation}

\subsubsection*{Bobinler Için Manyetik Alanın Şiddeti}
\begin{equation}
    B = 4\pi K  N \frac{I}{l} \qquad \qquad \parbox{4cm}{\footnotesize$\begin{aligned}
        B\text{: Manyetik Alan (T) } \\
        K\text{: Manyetik Alan Sabiti } \\
        N\text{: Sarım Sayısı } \\
        I\text{: Akım } (A) \\
        l\text{: Uzunluk (m) }
\end{aligned}$}
\end{equation}

\subsubsection*{Manyetik Alanın Tele Uyguladığı Kuvvet}
\begin{equation}
    F = B i l \qquad \qquad \parbox{4cm}{\footnotesize$\begin{aligned}
        F\text{: Kuvvet } (N) \\
        B\text{: Manyetik Alan (T) } \\
        i\text{: Akım } (A) \\
        l\text{: Telin Uzunluğu}
\end{aligned}$}
\end{equation}

\subsubsection*{Akım Geçen Tellerin Birbirine Uyguladığı Kuvvet}
\begin{equation}
    F = 2K \frac{i_1 i_2 l}{d} \qquad \qquad \parbox{4cm}{\footnotesize$\begin{aligned}
        F\text{: Kuvvet } (N) \\
        K\text{: Manyetik Alan Sabiti } \\
        i_1\text{, }i_2\text{: Tellerden Geçen Akımlar } (A) \\
        l\text{: Tellerin Uzunluğu} \\
        d\text{: Teller Arası Uzaklık (m) }
\end{aligned}$}
\end{equation}

\subsubsection*{Düzgün Manyetik Alanın Levha Üzerinde Oluşturduğu Tork}
\begin{equation}
    T = B i A \qquad \qquad \parbox{4cm}{\footnotesize$\begin{aligned}
        T\text{: Tork } (Nm) \\
        B\text{: Manyetik Alan (T) } \\
        i\text{: Akım } (A) \\
        A\text{: Levhanın Alanı} \\
\end{aligned}$}
\end{equation}

\subsubsection*{Manyetik Alanda Cismin Levha Dönme Yarıçapı}
\begin{equation}
    r = \frac{m v}{q B} \qquad \qquad \parbox{4cm}{\footnotesize$\begin{aligned}
        r\text{: Dönme Yarıçapı } (m) \\
        m\text{: Kütle } (kg) \\
        v\text{: Hız } (\frac{m}{s}) \\
        q\text{: Cismin Yükü } (C) \\
        B\text{: Manyetik Alan (T) }
\end{aligned}$}
\end{equation}
