\subsection{Sığaçlar}

\subsubsection*{Sığacın Sığası}
\begin{equation}
    C = \varepsilon \frac{A}{d} = \frac{q}{V} \qquad \qquad \parbox{4cm}{\footnotesize$\begin{aligned}
        C\text{: Sığa } \\
        E\text{: Elektrik Alan } \\
        A\text{: Levhanın Alanı} \\
        d\text{: Levhanın Uzaklığı } \\
        q\text{: Levhanın Yükü } \\
        V\text{: Levhanın Potansiyeli }
\end{aligned}$}
\end{equation}

\subsubsection*{Sığaçta Depolanan Enerji}
\begin{equation}
    E  = \frac{1}{2} qV = \frac{1}{2} C V^2 = \frac{q^2}{2C} \qquad \qquad \parbox{4cm}{\footnotesize$\begin{aligned}
        E\text{: Enerji (j) } \\
        C\text{: Sığa } \\
        q\text{: Levhanın Yükü } \\
        V\text{: Levhanın Potansiyeli }
\end{aligned}$}
\end{equation}
