\documentclass[./movement.tex]{subfiles}
\subsection{İvme}

İvme, birim zamandaki hız değişimini ifade eder.

\subsubsection*{Kütle Kuvvet İvme İlişkisi}

Kütle zamanla değişmiyorsa,

\begin{equation}
    \vec{a} = \frac{\vec{F}}{m} \qquad \qquad \parbox{4cm}{\footnotesize$\begin{aligned}
        \vec{a}\text{: İvme } (\frac{m}{s^2}) \\
        \vec{F}\text{: Kuvvet (N) } \\
        m\text{: Kütle (kg) }
    \end{aligned}$}
\end{equation}

olarak yazılabilir.

\subsubsection*{Sabit İvmeli Cisimler İçin İvme Hız İlişkisi}
\begin{equation}
    \vec{a} = \frac{\vec{v} - \vec{v_0}}{t} \qquad \qquad \parbox{4cm}{\footnotesize$\begin{aligned}
        \vec{a}\text{: İvme } (\frac{m}{s^2}) \\
        \vec{v}\text{: Hız } (\frac{m}{s}) \\
        \vec{v_0}\text{: Başlangıç Hızı } (\frac{m}{s}) \\
        t\text{: Zaman (s)} 
\end{aligned}$}
\end{equation}