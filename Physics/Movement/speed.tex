\documentclass[./movemet.tex]{subfiles}
\subsection{Hız}

Hız bir cismin birim zamanda aldığı yolu ifade eder.

\subsubsection*{Sabit Hızlı Cisimler İçin Hız Konum İlişkisi}
\begin{equation}
    \vec{v} = \frac{\vec{x} - \vec{x_0}}{t} \qquad \qquad \parbox{4cm}{\footnotesize$\begin{aligned}
        \vec{v}\text{: Hız } (\frac{m}{s}) \\
        \vec{x}\text{: Konum (m)} \\
        \vec{x_0}\text{: Başlangıç Konumu (m)} \\
        t\text{: Zaman (s)}
\end{aligned}$}
\end{equation}

\subsubsection*{Sabit İvmeli Cisimler İçin Hız Konum İlişkisi}
\begin{equation}
    \vec{v} = \vec{v_0} + \vec{a}t \qquad \qquad \parbox{4cm}{\footnotesize$\begin{aligned}
        \vec{v}\text{: Hız } (\frac{m}{s}) \\
        \vec{v_0}\text{: Başlangıç Hızı } (\frac{m}{s}) \\
        \vec{a}\text{: İvme } (\frac{m}{s^2}) \\
        t\text{: Zaman (s)}
\end{aligned}$}
\end{equation}

\subsubsection*{Tek Boyutta Sabit İvmeli Hareket İçin Hız İvme İlişkisi}
\begin{equation}
    v^2 = v_0 ^2 + 2ax \qquad \qquad \parbox{4cm}{\footnotesize$\begin{aligned}
        v\text{: Hız } (\frac{m}{s}) \\
        v_0\text{: Başlangıç Hızı } (\frac{m}{s}) \\
        a\text{: İvme } (\frac{m}{s^2}) \\
        x\text{: Konum Değişimi (m)}
\end{aligned}$}
\end{equation}

\subsubsection*{Limit Hız}

Küçük hızlarda limit hız,

\begin{equation}
    v_{limit} = \sqrt{\frac{mg}{kA}} \qquad \qquad \parbox{4cm}{\footnotesize$\begin{aligned}
        v_limit\text{: Limit Hız } (\frac{m}{s}) \\
        m\text{: Kütle (kg) } \\
        g\text{: Yerçekimi İvmesi} (\frac{m}{s^2}) \\
        k\text{: Sürtünme Katsayısı} \\ 
        A\text{: Yüzey Alanı } (m^2)
\end{aligned}$}
\end{equation}

şeklinde yazılabilir.

