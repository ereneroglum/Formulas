\subsection{Konum}

Konum vektörü bir cismin t anında bulunduğu yeri verir.

\subsubsection*{Sabit Hızlı Cisimler İçin Konum İlişkisi}
\begin{equation}
    \vec{x} = \vec{x_0} + \vec{v}t \qquad \qquad \parbox{4cm}{\footnotesize$\begin{aligned}
        \vec{x}\text{: Konum (m)} \\
        \vec{x_0}\text{: Başlangıç Konumu (m)} \\
        \vec{v}\text{: Hız } (\frac{m}{s}) \\
        t\text{: Zaman (s)}
\end{aligned}$}
\end{equation}

\subsubsection*{Sabit İvmeli Cisimler İçin Konum İlişkisi}
\begin{equation}
    \vec{x} = \vec{x_0} + \vec{v_0}t + \frac{1}{2} \vec{a} t^2 \qquad \qquad \parbox{4cm}{\footnotesize$\begin{aligned}
        \vec{x}\text{: Konum (m)} \\
        \vec{x_0}\text{: Başlangıç Konumu (m)} \\
        \vec{v}\text{: Hız } (\frac{m}{s}) \\
        \vec{v_0}\text{: Başlangıç Hızı } (\frac{m}{s}) \\
        \vec{a}\text{: İvme } (\frac{m}{s^2}) \\
        t\text{: Zaman (s)} 
\end{aligned}$}
\end{equation}

\subsubsection*{Atışlar İçin Menzil}
\begin{equation}
    x = \frac{2}{g}V_x V_y = \frac{2}{g}V^2\sin{\alpha}\cos{\alpha} \qquad \qquad \parbox{4cm}{\footnotesize$\begin{aligned}
        \vec{x}\text{: Menzil (m)} \\
        V_x\text{: X Ekseninde Hız } (\frac{m}{s}) \\
        V_y\text{: Y Ekseninde Hız } (\frac{m}{s}) \\
        V\text{: Hız } (\frac{m}{s}) \\
        \alpha\text{: Atış Açısı }
\end{aligned}$}
\end{equation}