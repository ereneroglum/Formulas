\subsection{Alternatif Akım}

\subsubsection*{Periyor ve Frekans}
\begin{equation}
    Tf = 1 \qquad \qquad \parbox{5.5cm}{\footnotesize$\begin{aligned}
        T\text{: Periyot (s) } \\
        f\text{: Frekans } (\frac{1}{s})
\end{aligned}$}
\end{equation}

\subsubsection*{Açısal Hız}
\begin{equation}
    w  = \frac{2\pi}{T} = 2 \pi f \qquad \qquad \parbox{5.5cm}{\footnotesize$\begin{aligned}
        w\text{: Açısal Hız } (\frac{\text{radyan}}{s}) \\
        T\text{: Periyot (s) } \\
        f\text{: Frekans } (\frac{1}{s}) \\
        v\text{: Hız } (\frac{m}{s})
\end{aligned}$}
\end{equation}

\subsubsection*{Akım Şiddeti Formulü}
\begin{equation}
    I = I_{m} \sin{wt} \qquad \qquad \parbox{4cm}{\footnotesize$\begin{aligned}
        I\text{: Akım Şiddeti } (A) \\
        I_m\text{: Maksimum Akım Şiddeti } (A) \\
        w\text{: Açısal Hız } (\frac{\text{radyan}}{s}) \\
        t\text{: Zaman } (s)
\end{aligned}$}
\end{equation}

\subsubsection*{Etkin Akım Şiddeti Formulü}
\begin{equation}
    I_e = \frac{I_{m}}{\sqrt 2} \qquad \qquad \parbox{4cm}{\footnotesize$\begin{aligned}
        I_e\text{: Etkin Akım Şiddeti } (A) \\
        I_m\text{: Maksimum Akım Şiddeti } (A)
\end{aligned}$}
\end{equation}

\subsubsection*{Gerilim Formulü}
\begin{equation}
    V = V_{m} \sin{wt} \qquad \qquad \parbox{4cm}{\footnotesize$\begin{aligned}
        V\text{: Gerilim Şiddeti } (V) \\
        V_m\text{: Maksimum Gerilim Şiddeti } (V) \\
        w\text{: Açısal Hız } (\frac{\text{radyan}}{s}) \\
        t\text{: Zaman } (s)
\end{aligned}$}
\end{equation}

\subsubsection*{Etkin Gerilim Formulü}
\begin{equation}
    V_e = \frac{V_{m}}{\sqrt 2} \qquad \qquad \parbox{4cm}{\footnotesize$\begin{aligned}
        V_e\text{: Etkin Gerilim Şiddeti } (A) \\
        V_m\text{: Maksimum Gerilim Şiddeti } (A)
\end{aligned}$}
\end{equation}

\subsubsection*{Bobinler İçin İndüktans Formulü}
\begin{equation}
    X_L = WL \qquad \qquad \parbox{4cm}{\footnotesize$\begin{aligned}
        X_L\text{: İndüktans (H)} \\
        w\text{: Açısal Hız } (\frac{\text{radyan}}{s}) \\
        L\text{: Öz İndiksiyon Katsayısı (H)}
\end{aligned}$}
\end{equation}

\subsubsection*{Kondansatörler İçin İndüktans Formulü}
\begin{equation}
    X_C = \frac{1}{WC} \qquad \qquad \parbox{4cm}{\footnotesize$\begin{aligned}
        X_C\text{: İndüktans (H)} \\
        w\text{: Açısal Hız } (\frac{\text{radyan}}{s}) \\
        C\text{: Sığa }
\end{aligned}$}
\end{equation}

\subsubsection*{R-L-C Devrelerinde Empedans Formulü}
\begin{equation}
    Z^2 = R^2 + (X_L - X_C)^2\qquad \qquad \parbox{4cm}{\footnotesize$\begin{aligned}
        Z\text{: Empedans } (\Omega) \\
        R\text{: Direnç } (\Omega) \\
        X_L\text{: Bobin İndüktansı (H)} \\
        X_C\text{: Kondansatör İndüktansı (H)}
\end{aligned}$}
\end{equation}

\subsubsection*{Empedans Akım Gerilim İlişkisi}
\begin{equation}
  V_e = I_eZ\qquad \qquad \parbox{4cm}{\footnotesize$\begin{aligned}
      V_e\text{: Etkin Gerilim Şiddeti } (A) \\
      I_e\text{: Etkin Akım Şiddeti } (A) \\
      Z\text{: Empedans } (\Omega)
\end{aligned}$}
\end{equation}

\subsubsection*{Rezonans Frekansı Formulü}
\begin{equation}
  f = \frac{1}{2 \pi \sqrt{LC}}\qquad \qquad \parbox{4cm}{\footnotesize$\begin{aligned}
      f\text{: Frekans } (\frac{1}{s}) \\
      L\text{: Öz İndiksiyon Katsayısı (H)} \\
      C\text{: Sığa }
\end{aligned}$}
\end{equation}
