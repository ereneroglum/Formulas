\section{Elektrik}

\subsubsection*{Telin Direnci}
\begin{equation}
    R = \rho \frac{l}{S} \qquad \qquad \parbox{4cm}{\footnotesize$\begin{aligned}
        R\text{: Direnç } (\Omega) \\
        l\text{: Uzunluk } (m) \\
        S\text{: Kesit Alanı } (m^2)
\end{aligned}$}
\end{equation}

\subsubsection*{Elektrik Devrelerinde Güç Formülü}
\begin{equation}
    P = VI = I^2 R = \frac{V^2}{R} \qquad \qquad \parbox{4cm}{\footnotesize$\begin{aligned}
        E\text{: Enerji (j) } \\
        V\text{: Potansiyel Fark } (V) \\
        I\text{: Akım Şiddeti } (A) \\
        R\text{: Direnç } (\Omega)
\end{aligned}$}
\end{equation}

\subsubsection*{Elektrik Devrelerinde Harcanan Enerji}
\begin{equation}
    E = VIt = I^2Rt = \frac{V^2}{R}t \qquad \qquad \parbox{4cm}{\footnotesize$\begin{aligned}
        E\text{: Enerji (j) } \\
        V\text{: Potansiyel Fark } (V) \\
        I\text{: Akım Şiddeti } (A) \\
        t\text{: Zaman (s)} \\
        R\text{: Direnç } (\Omega)
\end{aligned}$}
\end{equation}

\subsubsection*{Kirchoff Kanunu}
\begin{equation*}
  \begin{aligned}
    \text{Her bir döngüde toplam potansiyel 0'dır. } \\
    \text{Bir noktaya gelen akımlar toplamı, çıkan akımlar toplamına eşittir. }
    \end{aligned}
\end{equation*}

\newpage
\subimport{./}{direct.tex}
\newpage
\subimport{./}{alternative.tex}
