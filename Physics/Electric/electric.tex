\section{Elektrik}

\subsubsection*{Akım Şiddeti Formulü}
\begin{equation}
    I = \frac{q}{t} \qquad \qquad \parbox{4cm}{\footnotesize$\begin{aligned}
        I\text{: Akım Şiddeti } (A) \\
        q\text{: Yük Miktarı } (C) \\
        t\text{: Zaman (s)}
\end{aligned}$}
\end{equation}

\subsubsection*{Ohm Kanunu}
\begin{equation}
    V = IR \qquad \qquad \parbox{4cm}{\footnotesize$\begin{aligned}
        V\text{: Potansiyel Fark } (V) \\
        I\text{: Akım Şiddeti } (A) \\
        R\text{: Direnç } (\Omega)
\end{aligned}$}
\end{equation}

\subsubsection*{Seri Bağlı Dirençler için Eşdeğer Direnç Formulü}
\begin{equation}
    R_E = R_1 + R_2 + \dots \qquad \qquad \parbox{4cm}{\footnotesize$\begin{aligned}
        R\text{: Eşdeğer Direnç } (\Omega) \\
        R_1\text{, }R_2\text{: Seri Bağlı Dirençler } (\Omega)
\end{aligned}$}
\end{equation}

\subsubsection*{Parelel Bağlı Dirençler için Eşdeğer Direnç Formulü}
\begin{equation}
    \frac{1}{R_E} = \frac{1}{R_1} + \frac{1}{R_2} + \dots \qquad \qquad \parbox{4cm}{\footnotesize$\begin{aligned}
        R\text{: Eşdeğer Direnç } (\Omega) \\
        R_1\text{, }R_2\text{: Parelel Bağlı Dirençler } (\Omega)
\end{aligned}$}
\end{equation}

\subsubsection*{Telin Direnci}
\begin{equation}
    R = \rho \frac{l}{S} \qquad \qquad \parbox{4cm}{\footnotesize$\begin{aligned}
        R\text{: Direnç } (\Omega) \\
        l\text{: Uzunluk } (m) \\
        S\text{: Kesit Alanı } (m^2)
\end{aligned}$}
\end{equation}

\subsubsection*{Noktasal Yükün Elektriksel Potansiyeli}
\begin{equation}
    V = k \frac{q}{d} \qquad \qquad \parbox{4cm}{\footnotesize$\begin{aligned}
        V\text{: Elektriksel Potansiyel } (V) \\
        k\text{: Coulumb Sabiti } \\
        q\text{: Cismin Yük Miktarı } (C) \\
        d\text{: Mesafe (m) }
\end{aligned}$}
\end{equation}

\subimport{./}{capacitors.tex}
