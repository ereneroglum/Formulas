\subsubsection{Dış Açıortay Teoremi}
\begin{figure}[h!]
    \centering
    \begin{tikzpicture}
        \tkzDefPoint(40:5){H}
        \tkzDefPoint(40:4){A}
        \tkzDefPoint(0,0){B}
        \tkzDefPoint(3.5,0){C}
        \tkzDefLine[bisector](C,A,H)
        \tkzGetPoint{K}
        \tkzInterLL(B,C)(A,K)
        \tkzGetPoint{N}

        \tkzDrawPoints(A,B,C,N)
        \tkzDrawPolygon(A,B,C)
        \tkzDrawSegment(A,H)
        \tkzDrawSegment(A,N)
        \tkzDrawSegment(B,N)

        \tkzMarkAngle[size=0.5, mark=none](C,A,N)
        \tkzMarkAngle[size=0.5, mark=none](N,A,H)
        
        \tkzLabelAngle[pos=0.3](C,A,N){$\cdot$}
        \tkzLabelAngle[pos=0.3](N,A,H){$\cdot$}

        \tkzLabelPoints[below](C,N)
        \tkzLabelPoints[above left](A)
        \tkzLabelPoints[below left](B)

        \tkzLabelSegment[left](A,B){c}
        \tkzLabelSegment[left](A,C){b}

        \tkzLabelSegment[below](B,C){k}
        \tkzLabelSegment[below](C,N){p}
        
        \tkzLabelSegment[above right](A,N){$n_a$}


    \end{tikzpicture}
    \caption{Dış Açıortay}
    \label{fig:outerbisectrig}
\end{figure}

\ref{fig:bisect} Dış açıortay figüründe de görülebileceği gibi, 
\begin{equation}
    \begin{aligned}
    \frac{p}{p+k}=\frac{b}{c} \\
    n_a ^2 = p(p+k) - bc \\  
    \end{aligned}
\end{equation}
geçerlidir.
