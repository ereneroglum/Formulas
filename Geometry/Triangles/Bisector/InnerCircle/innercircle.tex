\subsubsection{İç Teğet Çember}
\begin{figure}[h!]
    \centering
    \begin{tikzpicture}
        \tkzDefPoint(-4,-1){B}
        \tkzDefPoint(0,4){A}
        \tkzDefPoint(2,0){C}
        \tkzLabelPoints[below left](B)
        \tkzLabelPoints[right](C)
        \tkzLabelPoints[above](A)
        \tkzDrawPolygon(A,B,C)
        \tkzLabelSegment[above left](A,B){c}
        \tkzLabelSegment(C,B){a}
        \tkzLabelSegment(A,C){b}

        \tkzDefTriangleCenter[in](A,B,C)
        \tkzGetPoint{I}
        \tkzLabelPoints[above right](I)
        \tkzDrawPoints(I)

        \tkzDefPointBy[projection=onto A--B](I) \tkzGetPoint{K}

        \tkzDrawCircle(I,K)

        \tkzMarkAngle[mark=none, arc=ll](I,B,A)
        \tkzMarkAngle[mark=none, arc=ll](C,B,I)

        \tkzMarkAngle(B,A,I)
        \tkzMarkAngle(I,A,C)

        \tkzMarkAngle[mark=none, size=0.4cm](A,C,I)
        \tkzMarkAngle[mark=none, size=0.4cm](I,C,B)
        \tkzLabelAngles[pos=0.25](A,C,I I,C,B){$\cdot$}

        \tkzDrawSegments(A,I B,I C,I)

    \end{tikzpicture}
    \caption{İç Teğet Çember}
    \label{fig:innercircle}
\end{figure}

\begin{equation}
    \text{\ref{fig:innercircle} Üçgen figüründe de görülebileceği gibi iç açıortaylar iç teğet çemberin merkezinde kesişir.}
\end{equation}