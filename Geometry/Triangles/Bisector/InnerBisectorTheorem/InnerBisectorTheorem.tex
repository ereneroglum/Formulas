\documentclass[../Bisector.tex]{subfiles}
\subsubsection{İç Açıortay Teoremi}
\begin{figure}[h!]
    \centering
    \begin{tikzpicture}
        \tkzDefPoint(45:3){A}
        \tkzDefPoint(0:0){B}
        \tkzDefPoint(0:5){C}
        \tkzDrawPolygon(A,B,C)
        \tkzDefLine[bisector](B,A,C) \tkzGetPoint{K}
        \tkzInterLL(A,K)(B,C)
        \tkzGetPoint{D}
        \tkzDrawSegment(A,D)
        \tkzDrawPoints(A,B,C,D)
        \tkzLabelPoints(B,C,D)
        \tkzLabelPoints[above](A)
        \tkzLabelSegments[below](B,D){$a_1$}
        \tkzLabelSegments[below](D,C){$a_2$}
        \tkzLabelSegments[above left](A,B){$c$}
        \tkzLabelSegments[above right](A,C){$b$}
        \tkzMarkAngle[size=0.5cm, color=black, mark=none](B,A,D)
        \tkzLabelAngle[pos=0.3](B,A,K){$\cdot$}
        \tkzMarkAngle[size=0.5cm, color=black, mark=none](D,A,C)
        \tkzLabelAngle[pos=0.3](D,A,C){$\cdot$}
        \tkzLabelSegments[left](A,D){$n_a$}
    \end{tikzpicture}
    \caption{İç Açıortay}
    \label{fig:bisect}
\end{figure}

\ref{fig:bisect} İç açıortay figüründe de görülebileceği gibi, 
\begin{equation}
    \begin{aligned}
    \frac{c}{a_1}=\frac{b}{a_2} \\
    n_a^2=bc-a_1 a_2 \\
    \end{aligned}
\end{equation}
geçerlidir.