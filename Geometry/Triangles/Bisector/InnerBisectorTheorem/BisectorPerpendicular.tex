\documentclass[../Bisector.tex]{subfiles}
\subsubsection{Açıortay Kollarına Atılan Dikme}
\begin{figure}[h!]
    \centering
    \begin{tikzpicture}
        \tkzDefPoint(5,4){B}
        \tkzDefPoint(0,2){A}
        \tkzDefPoint(5,0){C}

        \tkzDefLine[bisector](B,A,C) \tkzGetPoint{F}
        \tkzDefPointWith[linear,K=3](A,F) \tkzGetPoint{K}
        \tkzDefLine[perpendicular=through K](A,B) \tkzGetPoint{L}
        \tkzInterLL(K,L)(A,B) \tkzGetPoint{D}

        \tkzDefLine[perpendicular=through K](A,C) \tkzGetPoint{M}
        \tkzInterLL(K,M)(A,C) \tkzGetPoint{E}

        \tkzDrawSegments(B,A A,C K,D K,E A,K)
        \tkzDrawPoints(A,K,D,E)
        \tkzLabelPoints[left](A)
        \tkzLabelPoints[above left](D)
        \tkzLabelPoints[below left](E)
        \tkzLabelPoints[right](K)
        
        \tkzMarkAngle[size=0.5,mark=none](E,A,K)
        \tkzMarkAngle[size=0.5,mark=none](K,A,D)

        \tkzLabelAngle[pos=0.3](E,A,K){$\cdot$}
        \tkzLabelAngle[pos=0.3](K,A,D){$\cdot$}

        \tkzMarkSegments[mark=||](D,K K,E)
        \tkzMarkSegments[mark=|](D,A A,E)

        \tkzMarkRightAngle[size=0.3](A,D,K)
        \tkzMarkRightAngle[size=0.3](A,E,K)
        
        \tkzLabelAngle[pos=0.2](A,D,K){$\cdot$}
        \tkzLabelAngle[pos=-0.2](A,E,K){$\cdot$}

    \end{tikzpicture}
    \caption{Açıortay}
    \label{fig:bisector}
\end{figure}

\begin{equation}
    \text{\ref{fig:bisector} Açıortay figüründe de görülebileceği gibi açıortay ayaklarına atılan dikmelerin uzunlukları eşittir.}
\end{equation}