\subsubsection{İkizkenar Üçgende Tepeden Çizilen Dikme}
\begin{figure}[h!]
    \centering
    \begin{tikzpicture}
        \tkzDefPoint(-2.5,0){B}
        \tkzDefPoint(90:4){A}
        \tkzDefPoint(2.5,0){C}
        \tkzDrawPolygon(B,A,C)
        \tkzDefMidPoint(B,C)\tkzGetPoint{K}
        \tkzDrawSegment(A,K)
        \tkzMarkRightAngle[size=0.45, draw=black](B,K,A)
        \tkzLabelAngle[pos=-0.3](B,K,A){$\cdot$}
        \tkzLabelPoints(C)
        \tkzLabelPoints[below](K)
        \tkzLabelPoints[above](A)
        \tkzLabelPoints[below left](B)
        \tkzMarkSegment[mark=||](B,K)
        \tkzMarkSegment[mark=||](K,C)
        
        \tkzMarkSegment[mark=|](B,A)
        \tkzMarkSegment[mark=|](C,A)

        \tkzMarkAngle[size=0.5, mark=none](B,A,K)
        \tkzMarkAngle[size=0.5, mark=none](K,A,C)

        \tkzLabelAngle[pos=0.35](B,A,K){$\cdot$}
        \tkzLabelAngle[pos=0.35](K,A,C){$\cdot$}
    \end{tikzpicture}
    \caption{İkizkenar Üçgen}
    \label{fig:isosceles}
\end{figure}

\begin{equation}
    \text{İkizkenar üçgende tepeden çizilen dikme; yükseklik, kenarortay ve açıortaydır.}
\end{equation}