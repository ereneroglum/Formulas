\subsubsection{İkizkenar Üçgende Kesenin Uzunluğu}
\begin{figure}[h!]
    \centering
    \begin{tikzpicture}
        \tkzDefPoint(-2.5,0){B}
        \tkzDefPoint(90:4){A}
        \tkzDefPoint(2.5,0){C}
        \tkzDrawPolygon(B,A,C)
        \tkzDefPointOnLine[pos=0.3](B,C) \tkzGetPoint{K}
        \tkzDrawSegment(A,K)
        
        \tkzLabelPoints(C)
        \tkzLabelPoints[below](K)
        \tkzLabelPoints[above](A)
        \tkzLabelPoints[below left](B)
        \tkzLabelSegments[below](B,K){$m$}
        \tkzLabelSegments[below](K,C){$n$}
        
        \tkzMarkSegment[mark=||](B,A)
        \tkzMarkSegment[mark=||](C,A)

        \tkzLabelSegments[left=0.2cm](B,A){$a$}
        \tkzLabelSegments[right=0.2cm](C,A){$a$}

        \tkzLabelSegments[right](A,K){$d$}

        \tkzDrawPoints(A,B,C,K)
    \end{tikzpicture}
    \caption{İkizkenar Üçgen}
    \label{fig:isosceleswsecant}
\end{figure}

\ref{fig:isosceleswsecant} Üçgen figüründe de görülebileceği gibi,
\begin{equation} 
    d^2 = a^2 - mn
\end{equation}
geçerlidir.