\documentclass[../Isosceles.tex]{subfiles}
\subsubsection{İkizkenar Üçgende Kenarlara Atılan Dikmeler}
\begin{figure}[h!]
    \centering
    \begin{tikzpicture}
        \tkzDefPoint(0,0){A}
        \tkzDefPoint(6,0){C}
        \tkzDefMidPoint(A,C) \tkzGetPoint{D}
        \tkzDefLine[perpendicular=through D](A,C) \tkzGetPoint{K}
        \tkzDefPointWith[linear,K=0.8](D,K) \tkzGetPoint{B}

        \tkzDrawSegments(A,B B,C A,C)

        \tkzDefPoint(2,0){F}

        \tkzDefLine[perpendicular=through F](A,B) \tkzGetPoint{M}
        \tkzInterLL(F,M)(A,B) \tkzGetPoint{N}

        \tkzDrawSegment(F,N)
        \tkzMarkRightAngle(F,N,B)
        \tkzLabelAngle[pos=0.2](F,N,B){$\cdot$}

        \tkzDefLine[perpendicular=through F](B,C) \tkzGetPoint{O}
        \tkzInterLL(F,O)(B,C) \tkzGetPoint{P}

        \tkzDrawSegment(F,P)
        \tkzMarkRightAngle(B,P,F)
        \tkzLabelAngle[pos=0.2](B,P,F){$\cdot$}

        \tkzDrawPoints(A,B,C,F)
        \tkzLabelPoints[above](B)
        \tkzLabelPoints[above left](N)
        \tkzLabelPoints[above right](P)
        \tkzLabelPoints[below](A,F,C)
        
        \tkzMarkSegments[mark=||](B,A B,C)

        \tkzDrawSegment(B,D)
        \tkzMarkRightAngle(B,D,C)
        \tkzLabelAngle[pos=-0.2](B,D,C){$\cdot$}

        \tkzLabelSegment(B,D){$h$}


    \end{tikzpicture}
    \caption{İkizkenar Üçgen}
    \label{fig:isoscelesperptosegm}
\end{figure}

\ref{fig:isoscelesperptosegm} Üçgen figüründe de görülebileceği gibi, 
\begin{equation}
    \begin{aligned}
        h = \lvert FN \rvert + \lvert FP \rvert \\
    \end{aligned}
\end{equation}
geçerlidir.