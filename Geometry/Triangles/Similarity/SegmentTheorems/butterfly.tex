\documentclass[../Similarity.tex]{subfiles}
\subsubsection{Kelebek Benzerliği}
\begin{figure}[h!]
    \centering
    \begin{tikzpicture}
        \tkzDefPoint(0,6){A}
        \tkzDefPoint(6,6){B}
        \tkzDefPoint(2,0){D}
        \tkzDefPoint(8,0){E}
        \tkzInterLL(A,E)(B,D) \tkzGetPoint{C}

        \tkzDrawPolygon(A,B,C)
        \tkzDrawPolygon(D,E,C)
        
        \begin{scope}[decoration={
            markings,
            mark=at position 0.5 with {\arrow{stealth}}}
            ] 
            \tkzDrawSegments[postaction={decorate}](A,B)
            \tkzDrawSegments[postaction={decorate}](D,E)
        \end{scope}

        \tkzDrawPoints(A,B,C,D,E)

        \tkzLabelPoints[above left](A)
        \tkzLabelPoints[above right](B)
        \tkzLabelPoints[right=0.15](C)
        \tkzLabelPoints[below left](D)
        \tkzLabelPoints[below right](E)

        \tkzLabelSegments[above](A,B){$a$}
        \tkzLabelSegments[below](D,E){$d$}
        \tkzLabelSegments[below left](A,C){$b$}
        \tkzLabelSegments[below right](B,C){$c$}
        \tkzLabelSegments[above left](C,D){$f$}
        \tkzLabelSegments[above right](C,E){$e$}

        \tkzMarkAngles[mark=none, size=0.3cm, arc=ll](B,C,A D,C,E)
        \tkzMarkAngles[size=0.3cm](A,B,C E,D,C)
        \tkzMarkAngles[mark=none, size=0.4cm](C,A,B C,E,D)
        \tkzLabelAngles[pos=0.25](C,A,B C,E,D){$\cdot$}
    \end{tikzpicture}
    \caption{Temel Orantı Teoremi}
    \label{fig:ert}
\end{figure}

\ref{fig:ert} Figüründe de görülebileceği gibi,
\begin{equation}
    \begin{aligned}
        \frac{a}{d} = \frac{b}{e} = \frac{c}{f} \\
    \end{aligned}
\end{equation}
sağlanır.