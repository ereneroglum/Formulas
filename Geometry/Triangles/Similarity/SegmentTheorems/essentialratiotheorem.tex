\subsubsection{Temel Orantı Teoremi}
\begin{figure}[h!]
    \centering
    \begin{tikzpicture}
        \tkzDefPoint(75:6){A}
        \tkzDefPoint(0,0){B}
        \tkzDefPoint(6,0){C}
        \tkzDefMidPoint(B,A) \tkzGetPoint{D}
        \tkzDefMidPoint(C,A) \tkzGetPoint{E}

        \tkzLabelSegments[left=0.15cm](A,D){$a$}
        \tkzLabelSegments[left=0.15cm](D,B){$b$}
        \tkzLabelSegments[right=0.15cm](A,E){$c$}
        \tkzLabelSegments[right=0.15cm](E,C){$d$}
        \tkzLabelSegments[above](D,E){$e$}
        \tkzLabelSegments[above](B,C){$f$}
        
        \begin{scope}[decoration={
            markings,
            mark=at position 0.5 with {\arrow{stealth}}}
            ] 
            \tkzDrawSegments[postaction={decorate}](D,E)
            \tkzDrawSegments[postaction={decorate}](B,C)
        \end{scope}

        \tkzDrawPoints(A,B,C,D,E)
        \tkzDrawSegments(A,B)
        \tkzDrawSegments(A,C)
        \tkzLabelPoints[left](D,B)
        \tkzLabelPoints[right](C,E)
        \tkzLabelPoints[above](A)
    \end{tikzpicture}
    \caption{Temel Orantı Teoremi}
    \label{fig:ert}
\end{figure}

\ref{fig:ert} Üçgen figüründe de görülebileceği gibi,
\begin{equation}
    \begin{aligned}
        \frac{a}{a+b} = \frac{c}{c+d} = \frac{e}{f} \\
        \frac{a}{b} = \frac{c}{d}
    \end{aligned}
\end{equation}
sağlanır.