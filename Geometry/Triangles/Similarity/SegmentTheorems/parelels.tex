\subsubsection{Parelelleri Kesen Doğru İçin Benzerlik}
\begin{figure}[h!]
    \centering
    \begin{tikzpicture}
        \tkzDefPoint(0,0){A}
        \tkzDefPoint(60:5){B}
        \tkzDefPoint(4,0){C}
        \tkzDefPointWith[colinear=at C](A,B) \tkzGetPoint{D}
        \tkzDefPoint(7,0){E}
        \tkzDefPointWith[colinear=at E](A,B) \tkzGetPoint{F}
        \tkzInterLL(B,E)(F,A)
        \tkzGetPoint{G}
        \begin{scope}[decoration={
            markings,
            mark=at position 0.5 with {\arrow{>}}}
            ] 
            \tkzDrawSegments[postaction=decorate](A,B C,G E,F)
        \end{scope}
        \tkzDrawSegments(B,E F,A A,E)
        \tkzDrawPoints(A,B,C,G,E,F)

        \tkzLabelSegments[left=0.3cm](A,B){$a$}
        \tkzLabelSegments[left=0.2cm](C,G){$r$}
        \tkzLabelSegments[left=0.3cm](E,F){$b$}
    \end{tikzpicture}
    \caption{Parelelleri Kesen Doğru}
    \label{fig:lineeintersecparelels}
\end{figure}

\ref{fig:lineeintersecparelels} figüründe de görülebileceği gibi, 
\begin{equation}
    \frac{1}{r} = \frac{1}{a} + \frac{1}{b}
\end{equation}
geçerlidir.