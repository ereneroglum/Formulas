\subsubsection{Kenar Ortay Teoremi}
    
\begin{figure}[h!]
    \centering
    \begin{tikzpicture}
        \tkzDefPoint(45:3){A}
        \tkzDefPoint(0:0){B}
        \tkzDefPoint(0:5){C}
        \tkzDefMidPoint(B,C) \tkzGetPoint{K}
        \tkzDrawPolygon(A,B,C)
        \tkzDrawSegment(A,K)
        \tkzLabelSegment(A,K){$V_a$}
        \tkzDrawPoints(A,B,C,K)
        \tkzLabelPoints(B,C,K)
        \tkzLabelPoints[above](A)
        
        \tkzLabelSegments[above left](A,B){$c$}
        \tkzLabelSegments[above right](A,C){$b$}
        \tkzLabelSegments[below](B,C){$a$}
    \end{tikzpicture}
    \caption{Kenarortay}
    \label{fig:median}
\end{figure}

\ref{fig:median} Kenarortay figürü için,
\begin{equation}
    \begin{aligned}
        2 V_a^2 = b^2 + c^2 - \frac{a^2}{2}
    \end{aligned}
\end{equation}
geçerlidir. \par \vspace*{0.5cm}
Bütün üçgenler için,
\begin{equation}
    \begin{aligned}
    4(V_a^2 + V_b^2 + V_c^2) = 3(a^2 + b^2 + c^2) \\
    \end{aligned}
\end{equation}
geçerlidir.