\subsubsection{Apollonius Teoremi}
\begin{figure}[h!]
    \centering
    \begin{tikzpicture}
        \tkzDefPoint(60:4){A}
        \tkzDefPoint(0:0){B}
        \tkzDefPoint(0:5){C}
        \tkzDefMidPoint(B,C) \tkzGetPoint{K}
        \tkzDrawPolygon(A,B,C)
        \tkzDrawSegment(A,K)
        \tkzLabelSegment[right](A,K){$d$}
        \tkzDrawPoints(A,B,C,K)
        \tkzLabelPoints(B,C,K)
        \tkzLabelPoints[above](A)
        
        \tkzLabelSegments[above left](A,B){$c$}
        \tkzLabelSegments[above right](A,C){$b$}
        \tkzLabelSegments[below](B,K K,C){$m$}
    \end{tikzpicture}
    \caption{Kenarortayı Çizilmiş Üçgen}
    \label{fig:apoltrigwmedian}
\end{figure}

\ref{fig:apoltrigwmedian} Üçgen figüründe de görülebileceği gibi, 
\begin{equation}
    2(m^2 + d^2) = b^2 + c^2
\end{equation}
geçerlidir.