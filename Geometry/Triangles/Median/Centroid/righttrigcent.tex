\documentclass[./righttrigcent.tex]{subfiles}
\subsubsection{Dik Üçgende Kenarortay}
\begin{figure}[h!]
    \centering
    \begin{tikzpicture}
        \tkzDefPoint(0,0){A}
        \tkzDefPoint(0,3){B}
        \tkzDefPoint(2.5,0){C}

        \tkzDefTriangleCenter[centroid](A,B,C)
        \tkzGetPoint{G}

        \tkzDefMidPoint(B,C)
        \tkzGetPoint{D}

        \tkzDefMidPoint(A,C)
        \tkzGetPoint{E}

        \tkzDefMidPoint(A,B)
        \tkzGetPoint{F}

        \tkzDrawPolygon(A,B,C)
        \tkzDrawSegment(A,D)
        \tkzDrawSegment(B,E)
        \tkzDrawSegment(C,F)

        \tkzDrawPoints(A,B,C,D,G)

        \tkzLabelPoints[below left](A)
        \tkzLabelPoints[above](B)
        \tkzLabelPoints[below right](C)
        \tkzLabelPoints[above](G)

        \tkzMarkRightAngle[size=0.3](B,A,C)
        \tkzLabelAngle[pos=-0.2](B,A,C){$\cdot$}
    \end{tikzpicture}
    \caption{}
    \label{fig:righttrigmedian}
\end{figure}

\ref{fig:righttrigmedian} Üçgen figüründe de görülebileceği gibi G ağırlık merkezi ise,
\begin{equation}
    \begin{aligned}
        5V_a^2 = V_b^2 + V_c^2 \\
    \end{aligned}
\end{equation}
geçerlidir.