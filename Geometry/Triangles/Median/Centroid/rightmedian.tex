\documentclass[../centroid.tex]{subfiles}
\subsubsection{Dik Kesişen Kenarortay Uzunlukları}
\begin{figure}[h!]
    \centering
    \begin{tikzpicture}
        \tkzDefPoint(2,4){A}
        \tkzDefPoint(0,0){B}
        \tkzDefPoint(3.5,0){C}
        \tkzDrawPolygon(A,B,C)
        
        \tkzDefTriangleCenter[centroid](A,B,C)
        \tkzGetPoint{G}

        \tkzDefMidPoint(A,B)
        \tkzGetPoint{K}

        \tkzDefMidPoint(B,C)
        \tkzGetPoint{L}
        \tkzMarkSegments[mark=||](B,K K,A)

        \tkzDefMidPoint(A,C)
        \tkzGetPoint{M}
        \tkzMarkSegments[mark=|](A,M M,C)

        
        \tkzDrawSegments(B,M C,K)

        \tkzDrawPoints(A,B,C,G)

        \tkzLabelPoints[above](A)
        \tkzLabelPoints[below left](B)
        \tkzLabelPoints[below right](C)
        \tkzLabelPoints[above](G)

        \tkzMarkRightAngle(B,G,C)
        \tkzLabelAngle[pos=0.15](B,G,C){$\cdot$}

        \tkzLabelSegments[left](B,A){c}
        \tkzLabelSegments[right](C,A){b}
        \tkzLabelSegments[below](B,C){a}
    \end{tikzpicture}
    \caption{Üçgen}
    \label{fig:rightmed}
\end{figure}

\ref{fig:rightmed} Üçgen figüründe de görülebileceği gibi, 
\begin{equation}
    \begin{aligned}
        5a^2 = b^2 + c^2 \\
        V_a^2 = V_b^2 + V_c^2 \\
    \end{aligned}
\end{equation}
geçerlidir.