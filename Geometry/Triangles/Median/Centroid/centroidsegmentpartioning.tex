\documentclass[./centroid.tex]{subfiles}
\subsubsection{Ağırlık Merkezinde Kenar Uzunlukları Parçalama}
\begin{figure}[h!]
    \centering
    \begin{tikzpicture}
        \tkzDefPoint(2,4){A}
        \tkzDefPoint(0,0){B}
        \tkzDefPoint(5,0){C}
        \tkzDrawPolygon(A,B,C)
        
        \tkzDefTriangleCenter[centroid](A,B,C)
        \tkzGetPoint{G}


        \tkzDefMidPoint(B,C)
        \tkzGetPoint{L}

        \tkzDefMidPoint(A,B)
        \tkzGetPoint{K}

        \tkzDefMidPoint(A,C)
        \tkzGetPoint{M}


        \tkzDrawPoints(A,B,C,G)
        \tkzDrawSegments(A,L B,M C,K)

        \tkzLabelPoints[above](A)
        \tkzLabelPoints[below left](B)
        \tkzLabelPoints[below right](C)

        \tkzLabelPoints[above right](G)
        
        \tkzLabelSegments[right](A,G){$2a$}
        \tkzLabelSegments[right](G,L){$a$}

        \tkzLabelSegments[above](C,G){$2c$}
        \tkzLabelSegments[above](G,K){$c$}

        \tkzLabelSegments[above](B,G){$2b$}
        \tkzLabelSegments[above](G,M){$b$}


    \end{tikzpicture}
    \caption{Üçgen}
    \label{fig:centroidsegmentation}
\end{figure}

\begin{equation}
    \text{\ref{fig:centroidsegmentation} Üçgen figüründe de görülebileceği gibi G ağırlık merkezi ise G noktası kenarortayları} \frac{2}{1} \text{oranda böler.}
\end{equation}