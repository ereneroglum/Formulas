\subsubsection{312 Kuralı}
\begin{figure}[h!]
    \centering
    \begin{tikzpicture}
        \tkzDefPoint(2,4){A}
        \tkzDefPoint(0,0){B}
        \tkzDefPoint(5,0){C}
        \tkzDrawPolygon(A,B,C)
        
        \tkzDefTriangleCenter[centroid](A,B,C)
        \tkzGetPoint{G}

        \tkzDefMidPoint(A,B)
        \tkzGetPoint{D}
        \tkzMarkSegments[mark=|](A,D D,B)

        \tkzDefMidPoint(B,C)
        \tkzGetPoint{L}

        \tkzDefMidPoint(A,C)
        \tkzGetPoint{M}
        \tkzMarkSegments[mark=||](A,M M,C)

        \tkzDrawSegments(A,L)
        \tkzDrawSegments(B,M)
        \tkzDrawSegments(C,D)

        \tkzLabelPoints[above](A)
        \tkzLabelPoints[below left](B)
        \tkzLabelPoints[below right](C)
        \tkzLabelPoints[below right](G)

        \tkzDrawSegment(D,M)

        \tkzInterLL(A,L)(D,M)
        \tkzGetPoint{K}

        \tkzLabelSegments[right](A,K){$3k$}
        \tkzLabelSegments[right](K,G){$k$}
        \tkzLabelSegments[right](G,L){$2k$}

        \tkzDrawPoints(A,B,C,G,K)
        \tkzLabelPoints[above right](K)

        \tkzMarkSegments[mark=o](D,K K,M)

    \end{tikzpicture}
    \caption{}
    \label{fig:}
\end{figure}

\begin{equation}
    \text{\ref{fig:centroidsegmentation} Üçgen figüründe de görülebileceği gibi orta taban ve ağırlık merkezi, kenarortayı 3:1:2 oranında böler.}
\end{equation}