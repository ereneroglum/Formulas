\subsubsection{Ağırlık Merkezi}
\begin{figure}[h!]
    \centering
    \begin{tikzpicture}
        \tkzDefPoint(2,4){A}
        \tkzDefPoint(0,0){B}
        \tkzDefPoint(5,0){C}
        \tkzDrawPolygon(A,B,C)
        
        \tkzDefTriangleCenter[centroid](A,B,C)
        \tkzGetPoint{G}

        \tkzDefMidPoint(A,B)
        \tkzGetPoint{K}
        \tkzMarkSegments[mark=|](A,K K,B)

        \tkzDefMidPoint(B,C)
        \tkzGetPoint{L}
        \tkzMarkSegments[mark=||](B,L L,C)

        \tkzDefMidPoint(A,C)
        \tkzGetPoint{M}
        \tkzMarkSegments[mark=s](A,M M,C)

        
        \tkzDrawSegments(A,L B,M C,K)

        \tkzDrawPoints(A,B,C,G)

        \tkzLabelPoints[above](A)
        \tkzLabelPoints[below left](B)
        \tkzLabelPoints[below right](C)

        \tkzLabelPoints[above right](G)
    \end{tikzpicture}
    \caption{Üçgen}
    \label{fig:centeromasstrig}
\end{figure}

\begin{equation}
    \begin{aligned}
        \text{\ref{fig:centeromasstrig} Üçgen figüründe de görülebileceği gibi} \\
        \text{kenarortaylar ağırlık merkezinde kesişir.}
    \end{aligned}
\end{equation}
