\documentclass[../Median.tex]{subfiles}
\subsubsection{Dikten Atılan Kenarortay}
\begin{figure}[h!]
    \centering
    \begin{tikzpicture}
        \tkzDefPoint(0,0){A}
        \tkzDefPoint(0,5){B}
        \tkzDefPoint(4,0){C}

        \tkzDefMidPoint(B,C)
        \tkzGetPoint{D}

        \tkzDrawPolygon(A,B,C)
        \tkzDrawSegment(A,D)

        \tkzDrawPoints(A,B,C,D)

        \tkzLabelPoints[below left](A)
        \tkzLabelPoints[above](B)
        \tkzLabelPoints[below right](C)
        \tkzLabelPoints[above right](D)

        \tkzMarkSegment[mark=||](A,D)
        \tkzMarkSegment[mark=||](B,D)
        \tkzMarkSegment[mark=||](D,C)

        \tkzMarkRightAngle[size=0.3](B,A,C)
        \tkzLabelAngle[pos=-0.2](B,A,C){$\cdot$}
    \end{tikzpicture}
    \caption{Dik Üçgen}
    \label{fig:rightangwmedian}
\end{figure}

\begin{equation}
    \text{\ref{fig:rightangwmedian} Üçgen figüründe de görülebileceği gibi dikten atılan kenarortay uzunluğunun uzunluğu, böldüğü parçalara eşittir.}
\end{equation}