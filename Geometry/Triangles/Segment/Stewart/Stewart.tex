\subsubsection{Stewart Teoremi}
\begin{figure}[h!]
    \centering
    \begin{tikzpicture}
        \tkzDefPoint(60:4){A}
        \tkzDefPoint(0,0){B}
        \tkzDefPoint(5,0){C}
        \tkzDefPoint(3,0){D}

        \tkzDrawPolygon(A,B,C)
        \tkzDrawSegment(A,D)

        \tkzLabelPoints[below left](B)
        \tkzLabelPoints[above](A)
        \tkzLabelPoints[below right](C)
        \tkzLabelPoints[below](D)

        \tkzDrawPoints(A,B,C,D)

        \tkzLabelSegment[above left](A,B){c}
        \tkzLabelSegment[above right](A,C){b}
        \tkzLabelSegment[right](A,D){x}
        \tkzLabelSegment[below](B,D){m}
        \tkzLabelSegment[below](D,C){n}

    \end{tikzpicture}
    \caption{Üçgen}
    \label{fig:stewtrig}
\end{figure}

\ref{fig:stewtrig} Üçgen figüründe de görülebileceği gibi,
\begin{equation}
    \begin{aligned}
    x^2 = \frac{b^2 m + c^2 n}{m+n} - mn \\
    \end{aligned}
\end{equation}
geçerlidir.