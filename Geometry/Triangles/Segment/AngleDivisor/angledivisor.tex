\documentclass[../Segment.tex]{subfiles}
\subsubsection{Kesen Açı Kenar İlişkisi}
\begin{figure}[h!]
    \centering
    \begin{tikzpicture}
        \tkzDefPoint(45:3){A}
        \tkzDefPoint(0:0){B}
        \tkzDefPoint(0:5){C}
        \tkzDrawPolygon(A,B,C)
        \tkzDefPoint(2,0){D}
        \tkzDrawSegment(A,D)
        \tkzDrawPoints(A,B,C,D)
        \tkzLabelPoints(B,C,D)
        \tkzLabelPoints[above](A)
        \tkzLabelSegments[below](B,D){$a_1$}
        \tkzLabelSegments[below](D,C){$a_2$}
        \tkzLabelSegments[above left](A,B){$c$}
        \tkzLabelSegments[above right](A,C){$b$}
        \tkzMarkAngle[size=0.5cm, color=black, mark=none](B,A,D)
        \tkzLabelAngle[pos=0.7](B,A,D){$\alpha$}
        \tkzMarkAngle[size=0.5cm, color=black, mark=none](D,A,C)
        \tkzLabelAngle[pos=0.7](D,A,C){$\beta$}
    \end{tikzpicture}
    \caption{Üçgen}
    \label{fig:anglediv}
\end{figure}

\begin{equation}
    \text{\ref{fig:anglediv} Üçgen figüründe de görülebileceği gibi, } \\
    \frac{c}{a_1}\sin{\alpha} = \frac{b}{a_2}\sin{\beta} \\
    \text{ geçerlidir.}
\end{equation}