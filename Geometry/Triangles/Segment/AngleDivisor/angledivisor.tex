\subsubsection{Kesen Açı Kenar İlişkisi}
\begin{figure}[h!]
    \centering
    \begin{tikzpicture}
        \tkzDefPoint(45:3){A}
        \tkzDefPoint(0:0){B}
        \tkzDefPoint(0:5){C}
        \tkzDrawPolygon(A,B,C)
        \tkzDefPoint(2,0){D}
        \tkzDrawSegment(A,D)
        \tkzDrawPoints(A,B,C,D)
        \tkzLabelPoints(B,C,D)
        \tkzLabelPoints[above](A)
        \tkzLabelSegments[below](B,D){$a_1$}
        \tkzLabelSegments[below](D,C){$a_2$}
        \tkzLabelSegments[above left](A,B){$c$}
        \tkzLabelSegments[above right](A,C){$b$}
        \tkzMarkAngle[size=0.5cm, color=black, mark=none](B,A,D)
        \tkzLabelAngle[pos=0.7](B,A,D){$\alpha$}
        \tkzMarkAngle[size=0.5cm, color=black, mark=none](D,A,C)
        \tkzLabelAngle[pos=0.7](D,A,C){$\beta$}
    \end{tikzpicture}
    \caption{Üçgen}
    \label{fig:anglediv}
\end{figure}

\ref{fig:anglediv} Üçgen figüründe de görülebileceği gibi, 
\begin{equation}
    \begin{aligned}
    \frac{c}{a_1}\sin{\alpha} = \frac{b}{a_2}\sin{\beta} \\
    \end{aligned}
\end{equation}
geçerlidir.

\begin{figure}[h!]
    \centering
    \begin{tikzpicture}
        \tkzDefPoint(40:5){H}
        \tkzDefPoint(40:4){A}
        \tkzDefPoint(0,0){B}
        \tkzDefPoint(3.5,0){C}
        \tkzDefLine[bisector](C,A,H)
        \tkzGetPoint{K}
        \tkzInterLL(B,C)(A,K)
        \tkzGetPoint{N}

        \tkzDrawPoints(A,B,C,N)
        \tkzDrawPolygon(A,B,C)
        \tkzDrawSegment(A,H)
        \tkzDrawSegment(A,N)
        \tkzDrawSegment(B,N)

        \tkzMarkAngle[size=0.5, mark=none](C,A,N)
        \tkzMarkAngle[size=0.5, mark=none](N,A,H)
        
        \tkzLabelAngle[pos=0.8](N,A,H){$\theta _1$}
        \tkzLabelAngle[pos=0.8](C,A,N){$\theta _2$}

        \tkzLabelPoints[below](C,N)
        \tkzLabelPoints[above left](A)
        \tkzLabelPoints[below left](B)

        \tkzLabelSegment[left](A,B){c}
        \tkzLabelSegment[left](A,C){b}

        \tkzLabelSegment[below](B,C){k}
        \tkzLabelSegment[below](C,N){p}


    \end{tikzpicture}
    \caption{Üçgen}
    \label{fig:outeranglediv}
\end{figure}

\ref{fig:outeranglediv} Üçgen figüründe de görülebileceği gibi, 
\begin{equation}
    \begin{aligned}
    \frac{p}{p+k} \sin(\theta _1) =\frac{b}{c} \sin(\theta _2) \\ 
    \end{aligned}
\end{equation}
geçerlidir.
