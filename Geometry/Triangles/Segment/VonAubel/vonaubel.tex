\documentclass[../Segment.tex]{subfiles}
\subsubsection{Von Aubel Teoremleri}
\begin{figure}[h!]
    \centering
    \begin{tikzpicture}
        \tkzDefPoint(0,0){A}
        \tkzDefPoint(60:4){C}
        \tkzDefPoint(4,0){B}
        \tkzDefPoint(1.6,1.4){D}

        \tkzInterLL(A,C)(B,D) \tkzGetPoint{E}
        \tkzInterLL(C,B)(A,D) \tkzGetPoint{F}
        \tkzInterLL(B,A)(C,D) \tkzGetPoint{G}

        \tkzDrawSegment(B,E)
        \tkzDrawSegment(A,F)
        \tkzDrawSegment(C,G)

        \tkzDrawPolygon(A,B,C)

        \tkzDrawPoints(A,B,C,D,E,F,G)

        \tkzLabelPoints[below left](A){}
        \tkzLabelPoints[above](C)
        \tkzLabelPoints[below right](B)
        \tkzLabelPoints[right=0.2cm](D)
        \tkzLabelPoints[above left](E)
        \tkzLabelPoints[above right](F)
        \tkzLabelPoints[below](G)

        \tkzLabelSegment[above left](C,E){$b_1$}
        \tkzLabelSegment[above left](E,A){$b_2$}
        \tkzLabelSegment[above right](B,F){$a_2$}
        \tkzLabelSegment[above right](F,C){$a_1$}
        \tkzLabelSegment[right](C,D){$x_1$}
        \tkzLabelSegment[right](D,G){$x_2$}

    \end{tikzpicture}
    \caption{Üçgen}
    \label{fig:vonaubtrig}
\end{figure}

\begin{equation}
    \text{\ref{fig:vonaubtrig} Üçgen figüründe de görülebileceği gibi, } \\
    \frac{a_1}{a_2} + \frac{b_1}{b_2} = \frac{x_1}{x_2} \\
    \text{ geçerlidir.}
\end{equation}

\begin{figure}[h!]
    \centering
    \begin{tikzpicture}
        \tkzDefPoint(0,0){A}
        \tkzDefPoint(60:4){C}
        \tkzDefPoint(4,0){B}
        \tkzDefPoint(2.3,1){D}

        \tkzInterLL(A,C)(B,D) \tkzGetPoint{E}
        \tkzInterLL(C,B)(A,D) \tkzGetPoint{F}
        \tkzInterLL(B,A)(C,D) \tkzGetPoint{G}

        \tkzDrawSegment(B,E)
        \tkzDrawSegment(A,F)
        \tkzDrawSegment(C,G)

        \tkzDrawPolygon(A,B,C)

        \tkzDrawPoints(A,B,C,D,E,F,G)

        \tkzLabelPoints[below left](A){}
        \tkzLabelPoints[above](C)
        \tkzLabelPoints[below right](B)
        \tkzLabelPoints[right=0.2cm](D)
        \tkzLabelPoints[above left](E)
        \tkzLabelPoints[above right](F)
        \tkzLabelPoints[below](G)

        \tkzLabelSegment[above](B,D){$y_1$}
        \tkzLabelSegment[above](D,E){$y_2$}
        \tkzLabelSegment[above](A,D){$z_1$}
        \tkzLabelSegment[above](D,F){$z_2$}
        \tkzLabelSegment[right](C,D){$x_1$}
        \tkzLabelSegment[right](D,G){$x_2$}

    \end{tikzpicture}
    \caption{Üçgen}
    \label{fig:vonaubtrig2}
\end{figure}

\begin{equation}
    \text{\ref{fig:vonaubtrig2} Üçgen figüründe de görülebileceği gibi, } \\
    \frac{x_1}{x_1 + x_2} + \frac{y_1}{y_1 + y_2} + \frac{z_1}{z_1 + z_2} = 2 \\
    \frac{x_2}{x_1 + x_2} + \frac{y_2}{y_1 + y_2} + \frac{z_2}{z_1 + z_2} = 1 \\
    \text{ geçerlidir.}
\end{equation}