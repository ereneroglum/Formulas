\documentclass[../Segment.tex]{subfiles}
\subsubsection{Öklid Bağıntıları}
\begin{figure}[h!]
    \centering
    \begin{tikzpicture}
        \tkzDefPoint(0,0){A}
        \tkzDefPoint(0,5){B}
        \tkzDefPoint(4,0){C}

        \tkzDefLine[orthogonal=through A](B,C)
        \tkzGetPoint{K}
        \tkzInterLL(A,K)(B,C)
        \tkzGetPoint{D}

        \tkzDrawPolygon(A,B,C)
        \tkzDrawSegment(A,D)

        \tkzDrawPoints(A,B,C,D)
        \tkzMarkRightAngle(A,D,C)
        \tkzLabelAngle[pos=0.15](A,D,C){$\cdot$}

        \tkzMarkRightAngle(C,A,B)
        \tkzLabelAngle[pos=0.15](C,A,B){$\cdot$}

        \tkzLabelPoints[below left](A)
        \tkzLabelPoints[above](B)
        \tkzLabelPoints[below right](C)
        \tkzLabelPoints[above right](D)

        \tkzLabelSegments[left](A,B){c}
        \tkzLabelSegments[right](B,D){p}
        \tkzLabelSegments[right](D,C){k}
        \tkzLabelSegments[below](A,C){b}
        \tkzLabelSegments[below](A,D){h}

    \end{tikzpicture}
    \caption{Dik Üçgen}
    \label{fig:euclidtrig}
\end{figure}

\ref{fig:euclidtrig} Üçgen figüründe de görülebileceği gibi,
    \begin{equation}
        \begin{aligned}
            h^2 = pk \\
            b^2 = k (p+k) \\
            c^2 = p (p+k) \\
            \frac{1}{h^2}= \frac{1}{b^2} + \frac{1}{c^2} \\
        \end{aligned}
    \end{equation}
geçerlidir.
