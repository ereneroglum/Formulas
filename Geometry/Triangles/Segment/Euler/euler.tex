\subsubsection{Euler Teoremi}
\begin{figure}[h!]
    \centering
    \begin{tikzpicture}
        \tkzDefPoint(75:7){A}
        \tkzDefPoint(0,1){B}
        \tkzDefPoint(7,0){C}

        \tkzDrawPolygon(A,B,C)

        \tkzLabelPoints[above](A)
        \tkzLabelPoints[left](B)
        \tkzLabelPoints[right](C)

        \tkzDefTriangleCenter[circum](A,B,C)
        \tkzGetPoint{O}
        \tkzDefTriangleCenter[in](A,B,C)
        \tkzGetPoint{I}

        \tkzDefPointBy[projection=onto A--B](I) \tkzGetPoint{K}
        \tkzDrawCircle(I,K)
        \tkzDrawCircle(O,A)

        \tkzDrawPoints(A,B,C,I,O)

        \tkzLabelPoints[below left](I)
        \tkzLabelPoints[above right](O)

        \tkzDrawSegments(I,K O,C)
        \tkzLabelSegments[above](I,K){$r$}
        \tkzLabelSegments[right](O,C){$R$}

        \tkzDrawSegments[color=red](I,O)
        \tkzLabelSegments[above](O,I){$d$}

    \end{tikzpicture}
    \caption{İç  Teğet Çemberi ve Çevre Çemberi Çizilmiş Üçgen}
    \label{fig:eulertrig}
\end{figure}

\ref{fig:eulertrig} Üçgen figüründe de görülebileceği gibi I iç teğet çemberin merkezi ve O çevrel çemberin merkezi ise,
\begin{equation}
    \begin{aligned}
    d^2 = R(R-2r) \\
    \end{aligned}
\end{equation}
geçerlidir.