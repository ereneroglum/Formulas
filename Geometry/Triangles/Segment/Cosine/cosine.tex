\documentclass[../Segment.tex]{subfiles}
\subsubsection{Kosinüs Teoremi}
\begin{figure}[h!]
    \centering
    \begin{tikzpicture}
        \tkzDefPoint(0,0){B}
        \tkzDefPoint(45:4){A}
        \tkzDefPoint(7,0){C}

        \tkzDrawPolygon(B,A,C)
        \tkzMarkAngle[size=0.3, mark=none](B,A,C)
        \tkzLabelAngle[pos=0.5](B,A,C){$\alpha$}
        
        \tkzDrawPoints(B,A,C)
        \tkzLabelPoints[below left](B)
        \tkzLabelPoints[above](A)
        \tkzLabelPoints[below right](C)

        \tkzLabelSegment[below](B,C){$a$}
        \tkzLabelSegment[above right](A,C){$b$}
        \tkzLabelSegment[above left](A,B){$c$}

    \end{tikzpicture}
    \caption{Üçgen}
    \label{fig:costrig}
\end{figure}

\begin{equation}
    \text{\ref{fig:costrig} Üçgen figüründe de görülebileceği gibi, } \\
    a^2 = b^2 + c^2 - 2bc\cos{\alpha} \\
    \text{ geçerlidir.}
\end{equation}