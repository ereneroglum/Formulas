\documentclass[../Segment.tex]{subfiles}
\subsubsection{Melenaus Teoremi}
\begin{figure}[h!]
    \centering
    \begin{tikzpicture}
        \tkzDefPoint(0,0){A}
        \tkzDefPoint(60:3){B}
        \tkzDefPoint(2,0){C}
        \tkzDefPoint(5,0){D}
        \tkzDefMidPoint(B,C) \tkzGetPoint{E}
        \tkzDrawPolygon(A,B,C)
        \tkzDrawPolygon(C,D,E)
        \tkzInterLL(A,B)(D,E) \tkzGetPoint{F}
        \tkzDrawSegment(E,F)
        \tkzDrawPoints(A,B,C,D,E,F)

        \tkzLabelPoints[below left](A)
        \tkzLabelPoints[above](B)
        \tkzLabelPoints[below right](C)
        \tkzLabelPoints[below right](D)
        \tkzLabelPoints[above right](E)
        \tkzLabelPoints[above left](F)
        
    \end{tikzpicture}
    \caption{Üçgen}
    \label{fig:meltrig}
\end{figure}

\ref{fig:meltrig} Üçgen figüründe görülebileceği gibi, 
\begin{equation}
    \begin{aligned}
        \frac{\lvert CD \rvert}{\lvert AD \rvert} \frac{\lvert AF \rvert}{\lvert FB \rvert} \frac{\lvert BE \rvert}{\lvert EC \rvert} = 1 \\
        \frac{\lvert BF \rvert}{\lvert BA \rvert} \frac{\lvert AC \rvert}{\lvert CD \rvert} \frac{\lvert DE \rvert}{\lvert EF \rvert} = 1 \\
    \end{aligned}
\end{equation}
geçerlidir.