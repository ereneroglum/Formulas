\subsubsection{Steiner Teoremi}
\begin{figure}[h!]
	\centering
	\begin{tikzpicture}
		\tkzDefPoint(30:4){A}
		\tkzDefPoint(0,0){B}
		\tkzDefPoint(8,0){C}
		\tkzDefPoint(2,0){D}
		\tkzDefPoint(5,0){E}
		
		\tkzDrawPolygon(A,B,C)
		
		\tkzDrawPoints(A,B,C,D,E)
		\tkzDrawSegment(A,D)
		\tkzDrawSegment(A,E)
		
		\tkzLabelSegment[above left](A,B){c}
		\tkzLabelSegment[above right](A,C){b}
		\tkzLabelSegment[below](B,D){m}
		\tkzLabelSegment[below](D,E){p}
		\tkzLabelSegment[below](E,C){n}
		
		\tkzMarkAngle[mark=|](B,A,D)
		\tkzMarkAngle[mark=|](E,A,C)
		
		\tkzLabelPoints[above](A)
		\tkzLabelPoints[below left](B)
		\tkzLabelPoints[below right](C)
		\tkzLabelPoints[below](D)
		\tkzLabelPoints[below](E)
		
	\end{tikzpicture}
	\caption{Üçgen}
	\label{fig:steiner}
\end{figure}

\ref{fig:steiner} Üçgen figüründe de görülebileceği gibi
\begin{equation}
	\frac{c^2}{b^2} = \frac{m}{p+n}\frac{m+p}{n}
\end{equation}

