\documentclass[../Segment.tex]{subfiles}
\subsubsection{Pisagor Teoremi}
\begin{figure}[h!]
    \centering
    \begin{tikzpicture}
        \tkzDefPoint(0,0){A}
        \tkzDefPoint(0,5){B}
        \tkzDefPoint(4,0){C}

        \tkzDrawPolygon(A,B,C)

        \tkzDrawPoints(A,B,C)

        \tkzMarkRightAngle(C,A,B)
        \tkzLabelAngle[pos=0.15](C,A,B){$\cdot$}

        \tkzLabelPoints[below left](A)
        \tkzLabelPoints[above](B)
        \tkzLabelPoints[below right](C)

        \tkzLabelSegments[left](A,B){c}
        \tkzLabelSegments[right](B,C){a}
        \tkzLabelSegments[below](A,C){b}

    \end{tikzpicture}
    \caption{Dik Üçgen}
    \label{fig:pistrig}
\end{figure}

\begin{equation}
    \text{\ref{fig:pistrig} Üçgen figüründe de görülebileceği gibi, }
    a^2 = b^2 + c^2
    \text{ geçerlidir.}
\end{equation}