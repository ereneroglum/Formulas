\subsubsection{Ceva Teoremi}
\begin{figure}[h!]
    \centering
    \begin{tikzpicture}
        \tkzDefPoint(0,0){A}
        \tkzDefPoint(60:4){B}
        \tkzDefPoint(4,0){C}
        \tkzDefPoint(1.6,1.4){D}

        \tkzInterLL(A,B)(C,D) \tkzGetPoint{E}
        \tkzInterLL(B,C)(A,D) \tkzGetPoint{F}
        \tkzInterLL(C,A)(B,D) \tkzGetPoint{G}

        \tkzDrawSegment(C,E)
        \tkzDrawSegment(A,F)
        \tkzDrawSegment(B,G)

        \tkzDrawPolygon(A,B,C)

        \tkzDrawPoints(A,B,C,D,E,F,G)

        \tkzLabelPoints[below left](A){}
        \tkzLabelPoints[above](B)
        \tkzLabelPoints[below right](C)
        \tkzLabelPoints[right=0.2cm](D)
        \tkzLabelPoints[above left](E)
        \tkzLabelPoints[above right](F)
        \tkzLabelPoints[below](G)

        \tkzLabelSegment[below](A,G){$b_2$}
        \tkzLabelSegment[below](G,C){$b_1$}
        \tkzLabelSegment[above left](B,E){$c_2$}
        \tkzLabelSegment[above left](E,A){$c_1$}
        \tkzLabelSegment[above right](C,F){$a_2$}
        \tkzLabelSegment[above right](F,B){$a_1$}

    \end{tikzpicture}
    \caption{Üçgen}
    \label{fig:cevatrig}
\end{figure}

\ref{fig:cevatrig} Üçgen figüründe de görülebileceği gibi, 
\begin{equation}
    \begin{aligned}
    \frac{a_1}{a_2} \frac{b_1}{b_2} \frac{c_1}{c_2} = 1 \\
    \end{aligned}
\end{equation}
geçerlidir.