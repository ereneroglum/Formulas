\subsubsection{Sinüs Teoremi}
\begin{figure}[h!]
    \centering
    \begin{tikzpicture}
        \tkzDefPoint(0,0){A}
        \tkzDefPoint(45:4){B}
        \tkzDefPoint(7,0){C}

        \tkzDrawPolygon(A,B,C)
        \tkzMarkAngle[size=0.5, mark=none](C,A,B)
        \tkzLabelAngle[pos=0.7](C,A,B){$\alpha$}
        \tkzMarkAngle[size=0.5, mark=none](A,B,C)
        \tkzLabelAngle[pos=0.7](A,B,C){$\beta$}
        \tkzMarkAngle[size=0.5, mark=none](B,C,A)
        \tkzLabelAngle[pos=0.7](B,C,A){$\theta$}
        
        \tkzDrawPoints(A,B,C)
        \tkzLabelPoints[below left](A)
        \tkzLabelPoints[above](B)
        \tkzLabelPoints[below right](C)

        \tkzLabelSegment[below](A,C){$a$}
        \tkzLabelSegment[above right](B,C){$b$}
        \tkzLabelSegment[above left](B,A){$c$}

    \end{tikzpicture}
    \caption{Üçgen}
    \label{fig:sinetrig}
\end{figure}

\ref{fig:sinetrig} Üçgen figüründe de görülebileceği gibi, 
\begin{equation}
    \frac{a}{\sin{\alpha}} = \frac{b}{\sin{\beta}} = \frac{c}{\sin{\theta}} = 2R \qquad \qquad \parbox{4cm}{\footnotesize$\begin{aligned}
        R\text{: Çevrel Çemberin Yarıçapı}
    \end{aligned}$}
\end{equation}
geçerlidir.