\documentclass[../Segment.tex]{subfiles}
\subsubsection{Carnot Teoremi}
\begin{figure}[h!]
    \centering
    \begin{tikzpicture}
        \tkzDefPoint(60:4){A}
        \tkzDefPoint(0,0){B}
        \tkzDefPoint(4,0){C}
        \tkzDefPoint(1.5,1){D}
        
        \tkzDefLine[orthogonal=through D](B,A)
        \tkzGetPoint{K}
        \tkzDefLine[orthogonal=through D](A,C)
        \tkzGetPoint{L}
        \tkzDefLine[orthogonal=through D](B,C)
        \tkzGetPoint{M}

        \tkzInterLL(A,B)(K,D)
        \tkzGetPoint{E}
        \tkzInterLL(A,C)(L,D)
        \tkzGetPoint{F}
        \tkzInterLL(B,C)(M,D)
        \tkzGetPoint{G}

        \tkzDrawSegment(D,E)
        \tkzDrawSegment(D,F)
        \tkzDrawSegment(D,G)

        \tkzMarkRightAngle(D,E,A)
        \tkzMarkRightAngle(C,F,D)
        \tkzMarkRightAngle(D,G,B)


        \tkzLabelPoints[above](A)
        \tkzLabelPoints[below left](B)
        \tkzLabelPoints[below right](C)
        \tkzLabelPoints[above](D)
        \tkzLabelPoints[above left](E)
        \tkzLabelPoints[above right](F)
        \tkzLabelPoints[below](G)

        \tkzDrawPoints(A,B,C,D,E,F,G)

        \tkzDrawPolygon(A,B,C)

        \tkzLabelAngle[pos=0.17](D,E,A){$\cdot$}
        \tkzLabelAngle[pos=-0.20](C,F,D){$\cdot$}
        \tkzLabelAngle[pos=0.15](D,G,B){$\cdot$}

        \tkzLabelSegment[below](B,G){$a_1$}
        \tkzLabelSegment[below](G,C){$a_2$}
        \tkzLabelSegment[above right](A,F){$b_2$}
        \tkzLabelSegment[above right](F,C){$b_1$}
        \tkzLabelSegment[above left](A,E){$c_1$}
        \tkzLabelSegment[above left](E,B){$c_2$}

    \end{tikzpicture}
    \caption{Üçgen}
    \label{fig:carntrig}
\end{figure}

\ref{fig:carntrig} Üçgen figüründe görülebileceği gibi, 
\begin{equation}
    \begin{aligned}
    a_1 ^2 + b_1 ^2 + c_1 ^2 = a_2 ^2 + b_2 ^2 + c_2 ^2 \\
    \end{aligned}
\end{equation}
geçerlidir.