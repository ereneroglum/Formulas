\documentclass[../Area.tex]{subfiles}
\subsubsection{Üçgenin Alan Formulü}
\begin{figure}[h!]
    \centering
    \begin{tikzpicture}
        \tkzDefPoint(45:3){A}
        \tkzDefPoint(0,0){B}
        \tkzDefPoint(6,0){C}

        \tkzDrawPolygon(A,B,C)
        \tkzDrawPoints(A,B,C)
        
        \tkzLabelPoints[above](A)
        \tkzLabelPoints[below left](B)
        \tkzLabelPoints[below right](C)

        \tkzDefLine[perpendicular=through A](B,C) \tkzGetPoint{K}
        \tkzInterLL(A,K)(B,C) \tkzGetPoint{H}

        \tkzDrawSegment(A,H)

        \tkzMarkRightAngle[size=0.4](C,H,A)
        \tkzLabelAngle[pos=0.25](C,H,A){$\cdot$}

        \tkzLabelSegment[above left](A,B){c}
        \tkzLabelSegment[above right](A,C){b}
        \tkzLabelSegment[below](B,C){a}

        \tkzLabelSegment[right](A,H){h}

    \end{tikzpicture}
    \caption{Üçgen}
    \label{fig:trigwperpend}
\end{figure}

\begin{equation}
    \text{\ref{fig:trigwperpend} Üçgen figüründe de görülebileceği gibi, } \\
    Alan(ABC) = \frac{h*a}{2} \\
    \text{ geçerlidir.}
\end{equation}