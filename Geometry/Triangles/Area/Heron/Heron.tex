\documentclass[../Area.tex]{subfiles}
\subsubsection{Heron Teoremi}
\begin{figure}[h!]
    \centering
    \begin{tikzpicture}
        \tkzDefPoint(-2,-1){B}
        \tkzDefPoint(0,2){A}
        \tkzDefPoint(2,0){C}
        \tkzLabelPoints[below left](B)
        \tkzLabelPoints[right](C)
        \tkzLabelPoints[above](A)
        \tkzDrawPolygon(A,B,C)
        \tkzLabelSegment[above left](A,B){c}
        \tkzLabelSegment(C,B){a}
        \tkzLabelSegment(A,C){b}
    \end{tikzpicture}
    \caption{Üçgen}
    \label{fig:trig}
\end{figure}

\begin{equation}
    \text{\ref{fig:trig} Üçgen figüründe de görülebileceği gibi, } \\
    \text{Alan(ABC)} = \sqrt{u(u-a)(u-b)(u-c)} \qquad \qquad \parbox{4cm}{\footnotesize$\begin{aligned}
        u = \frac{a+b+c}{2}
    \end{aligned}$}
\end{equation}