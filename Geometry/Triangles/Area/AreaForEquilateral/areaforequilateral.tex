\subsubsection{Eşkenar Üçgenin Alan Formulü}
\begin{figure}[h!]
    \centering
    \begin{tikzpicture}
        \tkzDefPoint(0,0){A}
        \tkzDefPoint(5,0){B}
        \tkzDefPoint(60:5){C}

        \tkzDrawPoints(A,B,C)
        \tkzDrawPolygon(A,B,C)
        \tkzLabelSegments[below](A,B){$a$}
        \tkzLabelSegments[above left](A,C){$a$}
        \tkzLabelSegments[above right](B,C){$a$}
        
        \tkzMarkSegments[mark=||](A,B B,C A,C)
        \tkzLabelPoints[above](C)
        \tkzLabelPoints[below right](B)
        \tkzLabelPoints[below left](A)
    \end{tikzpicture}
    \caption{Eşkenar Üçgen}
    \label{fig:equilatforarea}
\end{figure}

\ref{fig:equilatforarea} Üçgen figüründede görülebileceği gibi,
\begin{equation}
    \text{Alan(ABC)} = \frac{a^2 \sqrt{3}}{4}
\end{equation} 
geçerlidir.