\subsubsection{İç Teğet Çember ile Alan}
\begin{figure}[h!]
    \centering
    \begin{tikzpicture}
        \tkzDefPoint(-4,-1){B}
        \tkzDefPoint(0,4){A}
        \tkzDefPoint(2,0){C}
        \tkzLabelPoints[below left](B)
        \tkzLabelPoints[right](C)
        \tkzLabelPoints[above](A)
        \tkzDrawPolygon(A,B,C)
        \tkzLabelSegment[above left](A,B){c}
        \tkzLabelSegment(C,B){a}
        \tkzLabelSegment(A,C){b}

        \tkzDefTriangleCenter[in](A,B,C)
        \tkzGetPoint{I}
        \tkzLabelPoints[above right](I)
        \tkzDrawPoints(I)

        \tkzDefPointBy[projection=onto A--B](I) \tkzGetPoint{K}
        \tkzDrawSegment(K,I)
        \tkzLabelSegments[above](K,I){$r$}

        \tkzDrawCircle(I,K)

    \end{tikzpicture}
    \caption{İç Teğet Çember ve Üçgen}
    \label{fig:innertrig}
\end{figure}


\ref{fig:innertrig} Üçgen figüründe de görülebileceği gibi I iç teğet çemberin merkezi ise,
\begin{equation}
    \text{Alan(ABC)} = ur \qquad \qquad \parbox{4cm}{\footnotesize$\begin{aligned}
        u = \frac{a+b+c}{2}
    \end{aligned}$}
\end{equation}
geçerlidir.