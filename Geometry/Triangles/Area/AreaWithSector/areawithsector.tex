\subsubsection{Doğru Parçasının Parçaladığı Alan}
\begin{figure}[h!]
    \centering
    \begin{tikzpicture}
        \tkzDefPoint(60:4){A}
        \tkzDefPoint(0,0){B}
        \tkzDefPoint(5,0){C}

        \tkzDefPointOnLine[pos=0.55](A,B)
        \tkzGetPoint{D}
        \tkzDefPointOnLine[pos=0.35](A,C)
        \tkzGetPoint{E}
        \tkzDrawPolygon(A,B,C)

        \tkzLabelPoints[below left](B)
        \tkzLabelPoints[above](A)
        \tkzLabelPoints[below right](C)
        \tkzLabelPoints[left](D)
        \tkzLabelPoints[right](E)

        \tkzDrawPoints(A,B,C,D,E)
        
        \tkzDrawSegment(E,D)

        \tkzLabelSegment[above left](A,D){$a$}
        \tkzLabelSegment[above left](D,B){$b$}

        \tkzLabelSegment[above right](A,E){$c$}
        \tkzLabelSegment[above right](E,C){$d$}

    \end{tikzpicture}
    \caption{}
    \label{fig:sectorareapartitioning}
\end{figure}

\ref{fig:sectorareapartitioning} Üçgen figüründe de görülebileceği gibi, 
\begin{equation}
    \begin{aligned}
    \frac{\text{Alan(ADE)}}{\text{Alan(ABC)}} = \frac{a}{a+b} \frac{c}{c+d} \\
    \end{aligned}
\end{equation}
geçerlidir.