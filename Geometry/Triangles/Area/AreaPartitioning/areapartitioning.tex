\documentclass[../Area.tex]{subfiles}
\subsubsection{Kesenin Parçaladığı Alan}
\begin{figure}[h!]
    \centering
    \begin{tikzpicture}
        \tkzDefPoint(60:4){A}
        \tkzDefPoint(0,0){B}
        \tkzDefPoint(5,0){C}
        \tkzDefPoint(3,0){D}

        \tkzDrawPolygon(A,B,C)
        \tkzDrawSegment(A,D)

        \tkzLabelPoints[below left](B)
        \tkzLabelPoints[above](A)
        \tkzLabelPoints[below right](C)
        \tkzLabelPoints[below](D)

        \tkzDrawPoints(A,B,C,D)

        \tkzLabelSegment[below](B,D){$k$}
        \tkzLabelSegment[below](D,C){$l$}

    \end{tikzpicture}
    \caption{}
    \label{fig:secareapartitioning}
\end{figure}

\ref{fig:secareapartitioning} Üçgen figüründe de görülebileceği gibi, 
\begin{equation}
    \begin{aligned}
    \frac{\text{Alan(ADB)}}{\text{Alan(ADC)}} = \frac{k}{l} \\
    \end{aligned}
\end{equation}
geçerlidir.