\subsubsection{Üçgen İçinde Üçgenin Parçaladığı Alan}
\begin{figure}[h!]
    \centering
    \begin{tikzpicture}
        \tkzDefPoint(60:4){A}
        \tkzDefPoint(0,0){B}
        \tkzDefPoint(5,0){C}

        \tkzDefPointOnLine[pos=0.45](A,B)
        \tkzGetPoint{D}
        \tkzDefPointOnLine[pos=0.65](A,C)
        \tkzGetPoint{E}
        \tkzDefPointOnLine[pos=0.25](B,C)
        \tkzGetPoint{F}
        \tkzDrawPolygon(A,B,C)

        \tkzLabelPoints[below left](B)
        \tkzLabelPoints[above](A)
        \tkzLabelPoints[below right](C)
        \tkzLabelPoints[left](D)
        \tkzLabelPoints[right](E)
        \tkzLabelPoints[below](F)

        \tkzDrawPoints(A,B,C,D,E,F)
        
        \tkzDrawPolygon(D,E,F)

        \tkzLabelSegment[above left](A,D){$a_1$}
        \tkzLabelSegment[above left](D,B){$a_2$}

        \tkzLabelSegment[above right](A,E){$c_2$}
        \tkzLabelSegment[above right](E,C){$c_1$}

        \tkzLabelSegment[below](B,F){$b_1$}
        \tkzLabelSegment[below](F,C){$b_2$}

    \end{tikzpicture}
    \caption{}
    \label{fig:triangleareapartitioning}
\end{figure}

\ref{fig:triangleareapartitioning} Üçgen figüründe de görülebileceği gibi, 
\begin{equation}
    \begin{aligned}
    \frac{\text{Alan(DEF)}}{\text{Alan(ABC)}} = \frac{a_1 b_1 c_1 + a_2 b_2 c_2}{(a_1 + a_2)(b_1 + b_2)(c_1 + c_2)} \\
    \end{aligned}
\end{equation}
geçerlidir.