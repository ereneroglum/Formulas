\subsubsection{Çevrel Çember ile Alan}
\begin{figure}[h!]
    \centering
    \begin{tikzpicture}
        \tkzDefPoint(-4,-1){B}
        \tkzDefPoint(0,4){A}
        \tkzDefPoint(2,0){C}
        \tkzLabelPoints[below left](B)
        \tkzLabelPoints[right](C)
        \tkzLabelPoints[above](A)
        \tkzDrawPolygon(A,B,C)
        \tkzLabelSegment[above left](A,B){$c$}
        \tkzLabelSegment(C,B){$a$}
        \tkzLabelSegment(A,C){$b$}

        \tkzDefTriangleCenter[circum](A,B,C)
        \tkzGetPoint{G}
        \tkzLabelPoints[below](G)
        \tkzDrawPoints(G)

        \tkzDrawSegment(C,G)
        \tkzLabelSegments[above](C,G){$R$}

        \tkzDrawCircle(G,A)

    \end{tikzpicture}
    \caption{Çevrel Çember ve Üçgen}
    \label{fig:circumtrig}
\end{figure}

\ref{fig:circumtrig} Üçgen figüründe de görülebileceği gibi G çevrel çemberin merkezi ise,
\begin{equation}
    \text{Alan(ABC)} = \frac{abc}{4R}
\end{equation}
geçerlidir.