\documentclass[../Area.tex]{subfiles}
\subsubsection{Sinüslü Alan Formulü}
\begin{figure}[h!]
    \centering
    \begin{tikzpicture}
        \tkzDefPoint(0,0){A}
        \tkzDefPoint(45:4){B}
        \tkzDefPoint(7,0){C}

        \tkzDrawPolygon(A,B,C)
        \tkzMarkAngle[size=0.5, mark=none](C,A,B)
        \tkzLabelAngle[pos=0.7](C,A,B){$\alpha$}
        
        \tkzDrawPoints(A,B,C)
        \tkzLabelPoints[below left](A)
        \tkzLabelPoints[above](B)
        \tkzLabelPoints[below right](C)

        \tkzLabelSegment[below](A,C){$a$}
        \tkzLabelSegment[above right](B,C){$b$}
        \tkzLabelSegment[above left](B,A){$c$}

    \end{tikzpicture}
    \caption{Üçgen}
    \label{fig:sineareatrig}
\end{figure}

\ref{fig:sineareatrig} Üçgen figüründe de görülebileceği gibi,11
\begin{equation}
    \begin{aligned}
    \text{Alan(ABC)} = \frac{1}{2}bc\sin{\alpha} \\
    \end{aligned}
\end{equation}
geçerlidir.