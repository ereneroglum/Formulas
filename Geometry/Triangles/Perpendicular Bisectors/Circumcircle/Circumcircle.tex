\subsubsection{Çevrel Çemberin Merkezi}
\begin{figure}[h!]
    \centering
    \begin{tikzpicture}
        \tkzDefPoint(-4,-1){B}
        \tkzDefPoint(0,4){A}
        \tkzDefPoint(2,0){C}
        \tkzLabelPoints[below left](B)
        \tkzLabelPoints[right](C)
        \tkzLabelPoints[above](A)
        \tkzDrawPolygon(A,B,C)
        
        \tkzDefTriangleCenter[circum](A,B,C)
        \tkzGetPoint{G}
        \tkzLabelPoints[below right](G)
        \tkzDrawPoints(G)

        \tkzDefPointBy[projection=onto A--B](G) \tkzGetPoint{K} 
        \tkzDefPointBy[projection=onto B--C](G) \tkzGetPoint{L}
        \tkzDefPointBy[projection=onto C--A](G) \tkzGetPoint{M}

        \tkzMarkRightAngle(G,K,A)
        \tkzMarkRightAngle(G,L,B)
        \tkzMarkRightAngle(G,M,C)
        
        \tkzLabelAngle[pos=0.2](G,K,A){$\cdot$}
        \tkzLabelAngle[pos=0.2](G,L,B){$\cdot$}
        \tkzLabelAngle[pos=0.2](G,M,C){$\cdot$}

        \tkzDrawPoints(A,B,C,K,L,M)

        \tkzMarkSegments[mark=|](A,K K,B)
        \tkzMarkSegments[mark=||](B,L L,C)
        \tkzMarkSegments[mark=s](C,M M,A)

        \tkzDrawSegments(G,K G,L G,M)
        \tkzDrawCircle(G,A)

    \end{tikzpicture}
    \caption{Çevrel Çember ve Üçgen}
    \label{fig:circumtrigwperp}
\end{figure}

\begin{equation}
    \begin{aligned}
        \text{\ref{fig:circumtrigwperp} Üçgen figüründe de görülebileceği gibi} \\
        \text{kenar orta dikmelerin kesişimi, çevrel çemberin merkezidir.}
    \end{aligned}
\end{equation}