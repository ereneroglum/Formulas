\subsubsection{Euler Teoremi}
\begin{figure}[h!]
    \centering
    \begin{tikzpicture}
        \tkzDefPoint(2,5){A}
        \tkzDefPoint(2,0){B}
        \tkzDefPoint(6,5){D}
        \tkzDefPoint(9,1){C}
        
        \tkzDrawPolygon(A,B,C,D)

        \tkzDrawSegments(A,C B,D)
        \tkzDefMidPoint(A,C) \tkzGetPoint{G}
        \tkzDefMidPoint(B,D) \tkzGetPoint{F}

        \tkzDrawSegments(G,F)
        \tkzDrawPoints(A,B,C,D,G,F)

        \tkzLabelSegments(G,F){$x$}
        \tkzLabelPoints[above left](A)
        \tkzLabelPoints[above right](D)
        \tkzLabelPoints[below left](B)
        \tkzLabelPoints[below right](C)
        \tkzLabelPoints[above](F)
        \tkzLabelPoints[above](G)
        
        \tkzLabelSegments[left](A,B){$a$}
        \tkzLabelSegments[below](B,C){$b$}
        \tkzLabelSegments[right](C,D){$c$}
        \tkzLabelSegments[above](D,A){$d$}

        \tkzLabelSegments[right=0.3cm](A,C){$e$}
        \tkzLabelSegments[left=0.3cm](B,D){$f$}

    \end{tikzpicture}
    \caption{Üçgen}
    \label{fig:eulerquad}
\end{figure}

\ref{fig:eulerquad} Dörtgen figüründe de görülebileceği gibi G ve F köşegenlerin orta noktası ise,  
\begin{equation}
    \begin{aligned}
    4 x^2 + e^2 + f^2 = a^2 + b^2 + c^2 + d^2
    \end{aligned}
\end{equation}
geçerlidir.