\subsubsection{Kirişler Dörgeninde Karşılıklı Açılar}
\begin{figure}[h!]
    \centering
    \begin{tikzpicture}
        \tkzDefPoint(4,4){I}
        \tkzDefPoint(0,4){P}

        \tkzDrawCircle(I,P)

        \tkzDefPoint(4,8){K}
        \tkzDefPoint(5,7.6){L}
        \tkzDefPoint(1,3){M}
        \tkzDefPoint(6,2){N}

        \tkzInterLC(K,L)(I,P) \tkzGetPoints{A}{B}
        \tkzInterLC(M,N)(I,P) \tkzGetPoints{C}{D}

        \tkzDrawPoints(A,B,C,D)

        \tkzDrawPolygon(A,B,C,D)
        
        \tkzMarkAngle[mark=none, size=0.5](A,B,C)
        \tkzLabelAngle[pos=0.7](A,B,C){$\alpha$}

        \tkzMarkAngle[mark=none, size=0.5](B,C,D)
        \tkzLabelAngle[pos=0.7](B,C,D){$\beta$}

        \tkzMarkAngle[mark=none, size=0.5](C,D,A)
        \tkzLabelAngle[pos=0.7](C,D,A){$\theta$}

        \tkzMarkAngle[mark=none, size=0.5](D,A,B)
        \tkzLabelAngle[pos=0.7](D,A,B){$\gamma$}

    \end{tikzpicture}
    \caption{Kirişler Dörtgeni}
    \label{fig:cyclicangles}
\end{figure}

\ref{fig:cyclicangles} Dörtgen figüründe de görülebileceği gibi, 
\begin{equation}
    \begin{aligned}
        \alpha + \theta = 180 ^{\circ} \\
        \beta + \gamma = 180 ^{\circ}
    \end{aligned}
\end{equation}
geçerlidir.