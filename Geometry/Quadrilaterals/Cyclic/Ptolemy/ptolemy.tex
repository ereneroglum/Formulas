\subsubsection{Batlamyus Teoremi}
\begin{figure}[h!]
    \centering
    \begin{tikzpicture}
        \tkzDefPoint(4,4){I}
        \tkzDefPoint(0,4){P}

        \tkzDrawCircle(I,P)

        \tkzDefPoint(4,8){K}
        \tkzDefPoint(5,7.6){L}
        \tkzDefPoint(1,3){M}
        \tkzDefPoint(6,2){N}

        \tkzInterLC(K,L)(I,P) \tkzGetPoints{A}{B}
        \tkzInterLC(M,N)(I,P) \tkzGetPoints{C}{D}

        \tkzDrawPoints(A,B,C,D)

        \tkzDrawPolygon(A,B,C,D)
        \tkzDrawSegments(A,C B,D)

        \tkzLabelSegment[above right](A,C){$e$}
        \tkzLabelSegment[above](B,D){$f$}

        \tkzLabelSegment[left](A,D){$a$}
        \tkzLabelSegment[below](A,B){$b$}
        \tkzLabelSegment[right](B,C){$c$}
        \tkzLabelSegment[below](C,D){$d$}

    \end{tikzpicture}
    \caption{Kirişler Dörgeni}
    \label{fig:ptolemycyclic}
\end{figure}

\ref{fig:ptolemycyclic} Dörtgen figüründe de görülebileceği gibi, 
\begin{equation}
    ac + bd = ef
\end{equation}
geçerlidir.