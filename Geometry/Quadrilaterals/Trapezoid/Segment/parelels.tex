\subsubsection{Yamukta Tabana Parelel Çekilen Kenar}
\begin{figure}[h!]
    \centering
    \begin{tikzpicture}
      \tkzDefPoint(0,0){A}
      \tkzDefPoint(7,0){B}
      \tkzDefPoint(2,5){C}
      \tkzDefPoint(5,5){D}

      \tkzDrawPolygon(A,C,D,B)

      \tkzDefPointOnLine[pos=0.4](A,C) \tkzGetPoint{K}
      \tkzDefPointOnLine[pos=0.4](B,D) \tkzGetPoint{L}

      \tkzDrawSegment(K,L)
      
      \tkzLabelSegments[above](C,D){$a$}
      \tkzLabelSegments[below](A,B){$b$}
      \tkzLabelSegments[above](K,L){$x$}
      
      \tkzLabelSegments[left](C,K){$c$}
      \tkzLabelSegments[left](A,K){$d$}

      \tkzDefPoint(3.5,4){S}
      \tkzDefPoint(3.5,1.5){W}

      \tkzLabelPoint(S){$S_1$}
      \tkzLabelPoint(W){$S_2$}
      \tkzDrawPoints(A,C,B,D,K,L)
      
    \end{tikzpicture}
    \caption{Yamuk}
    \label{fig:trapezoidparellel}
\end{figure}

\ref{fig:trapezoidparellel} Yamuk figüründe de görülebileceği gibi, 
\begin{equation}
  \begin{aligned}
    \frac{c}{d} = \frac{a-x}{x-b} \\
    \frac{a^2 - x^2}{x^2-b^2} = \frac{S_1}{S_2}
  \end{aligned}
\end{equation}
geçerlidir.
