\section{Asit Bazlar}

\subsubsection*{Asitler}
\begin{enumerate}
\item Suya $H^+$ veya $H_3O^+$ iyonu verirler.
\item Bazlarla tepkimeye girerek tuz oluştururlar.
\item Hidrojen sayıları tesir değerliklerini verir.
\item Turnusol kağıdını kırmızıya çevirirler.
\item Metallerle tepkimeye girerler. Metalin değerliği kadar H çıkarırlar.
\item Oksijenli olanları yarı soy metallerle (Ag, Cu, Hg) ile tepkimeye girerler.
\item Kral suyu ($3HCL + HNO_3$) tam soy metallerle (Pt, Au) tepkimeye girerler.
\end{enumerate}

\subsubsection*{Bazlar}
\begin{enumerate}
\item Suya $OH^-$ iyonu verirler.
\item Asitler ile tepkimeye girerek tuz oluştururlar.
\item $OH^-$ sayıları tesir değerliklerini verir.
\item Turnusol kağıdını maviye çevirirler.
\item 1A ve 2A metalleri (Be ve Mg hariç) suyla tepkimeye girerek baz oluşturur.
\item Kuvvetli bazlar amfoter metallerle (Zn, Pb, Cr, Be, Al, Sn) tepkimeye girerek tuz ve $H_2$ gazı ortaya çıkarırlar.
\end{enumerate}

\paragraph*{NOT}
Kuvvetli asitler ve bazlar tepkimeye girerse $H_2O$ oluştururlar.

\subsubsection*{pH Hesaplama}
\begin{equation}
    pH = -\log [H^+] \qquad \qquad \parbox{4cm}{\footnotesize$\begin{aligned}
        [H^+]\text{: $H^+$ derişimi }
\end{aligned}$}
\end{equation}
