\documentclass[./matter.tex]{subfiles}
\subsection{Mol}

Madde miktarına mol denir. Avagadro sayısı kadar atom veya molekül 1 mol'e karşılık gelir.

\subsubsection*{Mol Ağırlık İlişkisi}
\begin{equation}
    n = \frac{m}{M_a} \qquad \qquad \parbox{4cm}{\footnotesize$\begin{aligned}
        n\text{: Mol Sayısı } \\
        m\text{: Kütle (g) } \\
        M_a\text{: Molar Kütle (} \frac{g}{mol} \text{)} \\
\end{aligned}$}
\end{equation}

\subsubsection*{Mol Hacim İlişkisi}
\begin{equation}
    n = \frac{V}{V_0} \qquad \qquad \parbox{4cm}{\footnotesize$\begin{aligned}
        n\text{: Mol Sayısı }\\
        V\text{: Hacim } (m^3) \\
        V_0\text{: Normal Şartlarda Gazın Hacmi } (m^3)
\end{aligned}$}
\end{equation}


\subsubsection*{Mol Sayısı Atom İlişkisi}
\begin{equation}
    1 \textrm{ mol atom} = N_a \textrm{ tane atom} \qquad \qquad \parbox{4cm}{\footnotesize$\begin{aligned}
        N_a\text{: Avagadro Sayısı }
\end{aligned}$}
\end{equation}

