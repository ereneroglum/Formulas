\documentclass[./matter.tex]{subfiles}
\subsection{Koligatif Özellikler}

\subsubsection*{Kaynama Noktası Yükselme Formulü}
\begin{equation}
    T = k I M \qquad \qquad \parbox{4cm}{\footnotesize$\begin{aligned}
        T\text{: Sıcaklık Değişimi (\textdegree K)} \\
        k\text{: Kaynama Sabiti } \\
        I\text{: İyon Sayısı } \\
        M\text{: Molalite (} \frac{mol}{kg} \text{)} \\
\end{aligned}$}
\end{equation}

\subsubsection*{Denge Buhar Basıncı}
\begin{theorem}
    Denge buhar basıncı moleküller arası çekim kuvveti ile ters orantılıdır.
\end{theorem}