\subsection{Buhar Basıncı}

\subsubsection*{Homojen Karışımlarda Buhar Basıncı}
\begin{equation}
    P = \frac{n_s}{n_t} P_s \qquad \qquad \parbox{4cm}{\footnotesize$\begin{aligned}
        P\text{: Buhar Basıncı (Pa)} \\ 
        n_s\text{: Sıvının Mol Sayısı } \\
        n_t\text{: Toplam Mol Sayısı } \\
        P_s\text{: Normal Şartlarda Sıvının Buhar Basıncı (Pa)} \\ 
\end{aligned}$}
\end{equation}

\subsubsection*{Toplam Buhar Basıncı}
\begin{equation}
    P_t = P_1 + P_2 + \cdots \qquad \qquad \parbox{4cm}{\footnotesize$\begin{aligned}
        P_t\text{: Toplam Buhar Basıncı (Pa)} \\ 
        P_1\text{, }P_2\text{: Çözeltiyi Oluşturan Sıvıların Buhar Basıncı (Pa)}
\end{aligned}$}
\end{equation}

\subsubsection*{Buhar Basıncı}
\begin{theorem}
    Buhar basıncı moleküller arası çekim kuvveti ile ters orantılıdır.
\end{theorem}

\subsubsection*{Denge Buhar Basıncı}
\begin{theorem}
    Denge buhar basıncı moleküller arası çekim kuvveti ile ters orantılıdır.
\end{theorem}