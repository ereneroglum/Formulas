\subsection{Denge Tepkimeleri}

\begin{chemmath}
    aX + bY
    \reactrarrow{0pt}{1.5cm}{}{}
    cQ + dZ
  \end{chemmath}

\subsubsection*{Denge Sabiti}
\begin{equation}
K_c = \frac{[Q]^c [Z]^d }{[X]^a [Y]^b} = \frac{k_i}{k_g} \qquad \qquad \parbox{4cm}{\footnotesize$\begin{aligned}
    K_c\text{: Denge Sabiti} \\
    [Q]\text{: Q Derişimi (} \frac{mol}{L} \text{)} \\
    [Z]\text{: Z Derişimi (} \frac{mol}{L} \text{)} \\
    [X]\text{: X Derişimi (} \frac{mol}{L} \text{)} \\
    [Y]\text{: Y Derişimi (} \frac{mol}{L} \text{)} \\
    k_i\text{: İleri Tepkime Hızı } \\
    k_g\text{: Geri Tepkime Hızı }
\end{aligned}$}
\end{equation}

\begin{equation}
K_p = \frac{P_Q P_Z }{P_X P_Y} \qquad \qquad \parbox{4cm}{\footnotesize$\begin{aligned}
    K_p\text{: Kısmi Basınçlar Cinsi Denge Sabiti} \\
    P_Q\text{: Q'nun Basıncı} \text{ (Pa)} \\
    P_Z\text{: Z'nun Basıncı} \text{ (Pa)} \\
    P_X\text{: X'nun Basıncı} \text{ (Pa)} \\
    P_Y\text{: Y'nun Basıncı} \text{ (Pa)}
\end{aligned}$}
\end{equation}

\begin{equation}
K_p = K_c (RT)^{\delta n} \qquad \qquad \parbox{4cm}{\footnotesize$\begin{aligned}
    K_p\text{: Kısmi Basınçlar Cinsi Denge Sabiti} \\
    K_c\text{: Denge Sabiti } \\
    \delta n\text{: Ürünlerin Katsayısı - Girenlerin Katsayısı} \\
    R = \frac{22.4}{273} \\
    T\text{: Mutlak Sıcaklık (K)}
\end{aligned}$}
\end{equation}

\subsubsection*{Tepkimelerde İşlemler}
\begin{equation*}
  \begin{aligned}
    \text{Bir tepkime bir sayı n sayısı ile genişletiliyorsa, yeni denge sabiti eski denge sabitinin n üssü alınarak bulunur. } \\
    \text{İki tepkime toplanıyorsa, yeni denge sabiti, eski denge sabitlerinin çarpımıdır. } \\
    \text{Ters çevrilen tepkimenin denge sabiti, eski denge sabitinin -1 üssü alınarak bulunur. }
    \end{aligned}
\end{equation*}
