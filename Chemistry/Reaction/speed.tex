\subsection{Tepkime Hızı}

\begin{chemmath}
    aX_{\text{suda}} + bY_{\text{gaz}} + cZ_{\text{katı}}
    \reactrarrow{0pt}{1.5cm}{}{}
    dQ
  \end{chemmath}

\subsubsection*{Tepkime Hızı Formulü}
\begin{equation}
r = k [X]^a [Y]^b \qquad \qquad \parbox{4cm}{\footnotesize$\begin{aligned}
    r\text{: Tepkime Hızı} \\
    k\text{: Tepkime Katsayısı} \\
    [X]\text{: X Derişimi (} \frac{mol}{L} \text{)} \\
    [Y]\text{: Y Derişimi (} \frac{mol}{L} \text{)}
\end{aligned}$}
\end{equation}

\subsubsection*{Tepkime Derecesi Formulü}
\begin{equation}
n = a + b\qquad \qquad \parbox{4cm}{\footnotesize$\begin{aligned}
    n\text{: Tepkime Derecesi}
\end{aligned}$}
\end{equation}

\subsubsection*{Tepkime Katsayısı Formulü}
\begin{equation}
k = \frac{L^{n-1}}{mol^{n-1} s } \qquad \qquad \parbox{4cm}{\footnotesize$\begin{aligned}
    k\text{: Tepkime Katsayısı} \\
    n\text{: Tepkime Derecesi}
\end{aligned}$}
\end{equation}

\subsubsection*{Tepkime Molekülaritesi}
\begin{equation}
    m = a + b + c \qquad \qquad \parbox{4cm}{\footnotesize$\begin{aligned}
        m\text{: Tepkime Molekülaritesi}
    \end{aligned}$}
\end{equation}